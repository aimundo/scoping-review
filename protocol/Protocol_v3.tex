% Options for packages loaded elsewhere
\PassOptionsToPackage{unicode}{hyperref}
\PassOptionsToPackage{hyphens}{url}
\PassOptionsToPackage{dvipsnames,svgnames,x11names}{xcolor}
%
\documentclass[
]{article}

\usepackage{amsmath,amssymb}
\usepackage{lmodern}
\usepackage{iftex}
\ifPDFTeX
  \usepackage[T1]{fontenc}
  \usepackage[utf8]{inputenc}
  \usepackage{textcomp} % provide euro and other symbols
\else % if luatex or xetex
  \usepackage{unicode-math}
  \defaultfontfeatures{Scale=MatchLowercase}
  \defaultfontfeatures[\rmfamily]{Ligatures=TeX,Scale=1}
\fi
% Use upquote if available, for straight quotes in verbatim environments
\IfFileExists{upquote.sty}{\usepackage{upquote}}{}
\IfFileExists{microtype.sty}{% use microtype if available
  \usepackage[]{microtype}
  \UseMicrotypeSet[protrusion]{basicmath} % disable protrusion for tt fonts
}{}
\makeatletter
\@ifundefined{KOMAClassName}{% if non-KOMA class
  \IfFileExists{parskip.sty}{%
    \usepackage{parskip}
  }{% else
    \setlength{\parindent}{0pt}
    \setlength{\parskip}{6pt plus 2pt minus 1pt}}
}{% if KOMA class
  \KOMAoptions{parskip=half}}
\makeatother
\usepackage{xcolor}
\usepackage[top=20mm,left=20mm,heightrounded]{geometry}
\setlength{\emergencystretch}{3em} % prevent overfull lines
\setcounter{secnumdepth}{5}
% Make \paragraph and \subparagraph free-standing
\ifx\paragraph\undefined\else
  \let\oldparagraph\paragraph
  \renewcommand{\paragraph}[1]{\oldparagraph{#1}\mbox{}}
\fi
\ifx\subparagraph\undefined\else
  \let\oldsubparagraph\subparagraph
  \renewcommand{\subparagraph}[1]{\oldsubparagraph{#1}\mbox{}}
\fi


\providecommand{\tightlist}{%
  \setlength{\itemsep}{0pt}\setlength{\parskip}{0pt}}\usepackage{longtable,booktabs,array}
\usepackage{calc} % for calculating minipage widths
% Correct order of tables after \paragraph or \subparagraph
\usepackage{etoolbox}
\makeatletter
\patchcmd\longtable{\par}{\if@noskipsec\mbox{}\fi\par}{}{}
\makeatother
% Allow footnotes in longtable head/foot
\IfFileExists{footnotehyper.sty}{\usepackage{footnotehyper}}{\usepackage{footnote}}
\makesavenoteenv{longtable}
\usepackage{graphicx}
\makeatletter
\def\maxwidth{\ifdim\Gin@nat@width>\linewidth\linewidth\else\Gin@nat@width\fi}
\def\maxheight{\ifdim\Gin@nat@height>\textheight\textheight\else\Gin@nat@height\fi}
\makeatother
% Scale images if necessary, so that they will not overflow the page
% margins by default, and it is still possible to overwrite the defaults
% using explicit options in \includegraphics[width, height, ...]{}
\setkeys{Gin}{width=\maxwidth,height=\maxheight,keepaspectratio}
% Set default figure placement to htbp
\makeatletter
\def\fps@figure{htbp}
\makeatother
\newlength{\cslhangindent}
\setlength{\cslhangindent}{1.5em}
\newlength{\csllabelwidth}
\setlength{\csllabelwidth}{3em}
\newlength{\cslentryspacingunit} % times entry-spacing
\setlength{\cslentryspacingunit}{\parskip}
\newenvironment{CSLReferences}[2] % #1 hanging-ident, #2 entry spacing
 {% don't indent paragraphs
  \setlength{\parindent}{0pt}
  % turn on hanging indent if param 1 is 1
  \ifodd #1
  \let\oldpar\par
  \def\par{\hangindent=\cslhangindent\oldpar}
  \fi
  % set entry spacing
  \setlength{\parskip}{#2\cslentryspacingunit}
 }%
 {}
\usepackage{calc}
\newcommand{\CSLBlock}[1]{#1\hfill\break}
\newcommand{\CSLLeftMargin}[1]{\parbox[t]{\csllabelwidth}{#1}}
\newcommand{\CSLRightInline}[1]{\parbox[t]{\linewidth - \csllabelwidth}{#1}\break}
\newcommand{\CSLIndent}[1]{\hspace{\cslhangindent}#1}

\usepackage{lineno}
\usepackage{setspace}
\linenumbers
\doublespacing
\makeatletter
\makeatother
\makeatletter
\makeatother
\makeatletter
\@ifpackageloaded{caption}{}{\usepackage{caption}}
\AtBeginDocument{%
\ifdefined\contentsname
  \renewcommand*\contentsname{Table of contents}
\else
  \newcommand\contentsname{Table of contents}
\fi
\ifdefined\listfigurename
  \renewcommand*\listfigurename{List of Figures}
\else
  \newcommand\listfigurename{List of Figures}
\fi
\ifdefined\listtablename
  \renewcommand*\listtablename{List of Tables}
\else
  \newcommand\listtablename{List of Tables}
\fi
\ifdefined\figurename
  \renewcommand*\figurename{Figure}
\else
  \newcommand\figurename{Figure}
\fi
\ifdefined\tablename
  \renewcommand*\tablename{Table}
\else
  \newcommand\tablename{Table}
\fi
}
\@ifpackageloaded{float}{}{\usepackage{float}}
\floatstyle{ruled}
\@ifundefined{c@chapter}{\newfloat{codelisting}{h}{lop}}{\newfloat{codelisting}{h}{lop}[chapter]}
\floatname{codelisting}{Listing}
\newcommand*\listoflistings{\listof{codelisting}{List of Listings}}
\makeatother
\makeatletter
\@ifpackageloaded{caption}{}{\usepackage{caption}}
\@ifpackageloaded{subcaption}{}{\usepackage{subcaption}}
\makeatother
\makeatletter
\@ifpackageloaded{tcolorbox}{}{\usepackage[many]{tcolorbox}}
\makeatother
\makeatletter
\@ifundefined{shadecolor}{\definecolor{shadecolor}{rgb}{.97, .97, .97}}
\makeatother
\makeatletter
\makeatother
\ifLuaTeX
  \usepackage{selnolig}  % disable illegal ligatures
\fi
\IfFileExists{bookmark.sty}{\usepackage{bookmark}}{\usepackage{hyperref}}
\IfFileExists{xurl.sty}{\usepackage{xurl}}{} % add URL line breaks if available
\urlstyle{same} % disable monospaced font for URLs
\hypersetup{
  pdftitle={Scoping Review Protocol: Statistical Models for Longitudinal Data},
  pdfauthor={Ariel I. Mundo Ortiz},
  colorlinks=true,
  linkcolor={blue},
  filecolor={Maroon},
  citecolor={Blue},
  urlcolor={Blue},
  pdfcreator={LaTeX via pandoc}}

\title{Scoping Review Protocol: Statistical Models for Longitudinal
Data}
\author{Ariel I. Mundo Ortiz}
\date{2022-08-18}

\begin{document}
\maketitle
\ifdefined\Shaded\renewenvironment{Shaded}{\begin{tcolorbox}[sharp corners, enhanced, breakable, boxrule=0pt, frame hidden, interior hidden, borderline west={3pt}{0pt}{shadecolor}]}{\end{tcolorbox}}\fi

\renewcommand*\contentsname{Table of contents}
{
\hypersetup{linkcolor=}
\setcounter{tocdepth}{3}
\tableofcontents
}
\hypertarget{background}{%
\section{Background}\label{background}}

Longitudinal studies are frequently used in the health sciences
(biomedical research, epidemiology, public health, among others) as they
allow to examine how the temporal effect of a treatment or an
intervention, in contrast to a cross-sectional study, which only allows
to examine the effect of the intervention at a single time point. When
compared their cross-sectional counterparts, longitudinal studies allow
for increased statistical power and more cost efficient
strategies\textsuperscript{1,2}. However, the statistical analysis of
longitudinal data requires to take into consideration factors such as
data missingness, correlation, and non-linear trends, which do not occur
on cross-sectional data\textsuperscript{3,4}. In other words, there is
an ``analytic cost'' associated with the increased complexity of
longitudinal data\textsuperscript{2}.

This additional layer of complexity has led to a problem of model
misspecification in the statistical analysis of the data (i.e., the use
of a statistical model that is not coherent with the data), which has
been reported to occur in many fields, including the health
sciences\textsuperscript{5}. For example, in a landmark study Liu et
al.~showed that in a subset of papers in the biomedical sciences, the
most popular model used to analyze longitudinal data was the analysis of
variance (ANOVA, an approach that fails to take into account the
correlation between measures over time), and that only 18\% of the
studies analyzed used models intended for longitudinal analysis while
checking that the assumptions of the model were satisfied by the
data\textsuperscript{6}.

Historically, the repeated measures ANOVA (rm-ANOVA, a statistical model
for longitudinal data) has been the preferred method in the health
sciences to analyze longitudinal data, despite the fact that the
multiple assumptions required by this model are frequently not satisfied
by the data collected in longitudinal studies\textsuperscript{4}. On the
other hand, the last 30 years have seen incredible progress in the field
of Statistics with the development of statistical models for
longitudinal data that relax the assumptions of rm-ANOVA. Linear mixed
models, generalized additive models, Bayesian models, and generalized
estimating equations are among these modern statistical models developed
for longitudinal data\textsuperscript{7--11}. From these statistical
methods, linear mixed models and generalized estimating equations are
the two classes of models that have been frequently applied to analyze
longitudinal data in the health sciences during the last
decade\textsuperscript{12--14}.

However, modern statistical methods that are suited to analyze
longitudinal data have been the exception rather than the norm in the
health sciences. In 2001, a study reported that only 30\% of the
clinical trials analyzed used linear mixed models to analyze their
results, and that the preferred method of analysis continued to be
rm-ANOVA\textsuperscript{15} (in comparison, McCullagh and Nelder's
seminal book on the generalized linear model (GLM) was published in
1989\textsuperscript{16}, and there was ongoing work on the extension of
the GLM framework to the mixed model case by 1993\textsuperscript{17}).
Apart from the aforementioned study, there are not recent papers that
examine the use of modern statistical methods for longitudinal data in
the health sciences. Such information is critical to understand if the
use of these methods has increased or decreased in the field over the
last 20 years, and the reasons behind such changes.

Additionally, the reproducibility crisis is an ongoing issue in the
health sciences\textsuperscript{18,19}, a major component of it being
the misuse and lack of reproducibility of statistical
analyses\textsuperscript{20,21}. Despite the fact that the landscape of
statistical software has vastly increased in the last decade with many
statistical computational tools (software, packages) now available to
researchers, reproducibility standards vary between each computational
tool\textsuperscript{22}. Furthermore, there is still high variability
in the amount of statistical reporting across
journals\textsuperscript{23}. Understanding what statistical
computational tools are used nowadays by researchers in the health
sciences can provide an assessment of the advances in the field towards
research reproducibility, while identifying limitations that might still
be in place.

In this study, we surveyed the statistical methods used in papers
dealing with longitudinal data in the health sciences in order to: 1)
identify statistical methods used in order to assess the trends in
adoption of modern statistical methods, 2) determine what are the
computational tools used by researchers to perform statistical analyses,
and 3) use the previous points to provide context to the current status
of the advances in research reproducibility in the field.

\hypertarget{objective}{%
\section{Objective}\label{objective}}

This study aims to summarize the different statistical models for
longitudinal data that are used in the health sciences to identify the
current extent in the adoption of modern statistical methods, determine
what are the computational tools used in each case and how this in turn
affects the reproducibility, and provide an updated list on methods
recently developed for longitudinal data in order to determine if they
can be broadly applied to longitudinal data in the health sciences.

\hypertarget{review-question}{%
\section{Review Question}\label{review-question}}

Summarize the statistical methods used to analyze longitudinal data in
the health sciences to identify which methods are most commonly used,
the applicability of such methods in the context of each study, and gaps
that might exist that prevent the adoption of modern statistical methods
that can be better suited to analyze the data. Additionally, identify if
studies check for model assumptions, and how this in turn impacts the
reported results.

\hypertarget{databases}{%
\section{Databases}\label{databases}}

\begin{itemize}
\tightlist
\item
  PubMed
\item
  Web of Science
\end{itemize}

\hypertarget{search-terms}{%
\section{Search Terms}\label{search-terms}}

\hypertarget{for-the-application-of-modern-models-on-longitudinal-biomedicalhealth-data}{%
\subsection{For the Application of Modern Models on Longitudinal
Biomedical/Health
Data}\label{for-the-application-of-modern-models-on-longitudinal-biomedicalhealth-data}}

\hypertarget{pubmed}{%
\subsubsection{PubMed}\label{pubmed}}

\hypertarget{query-1}{%
\paragraph{Query 1:}\label{query-1}}

(biomedical OR health) AND ((repeated measures) OR (longitudinal study)
OR (ANOVA) OR (mixed effects) OR (growth curve) OR (generalized additive
model) OR (generalized estimating equation)) NOT ((review) OR (meta
analysis))

Hits: 393,188

Comments: query picks too many papers, and is not specific

\hypertarget{query-2}{%
\paragraph{Query 2:}\label{query-2}}

(biomedical OR health) AND ((repeated measures) OR (longitudinal study))
AND ((statistical analyses) OR (statistical analysis)) NOT ((review) OR
(meta analysis))

Hits: 12,617

Comments: \textcolor{blue}{This is the best query so far}.

Papers from this query appear to be good. The query catches many papers
from psychology and psychiatry, but the ones I checked did said used
linear mixed models or regression in their analyses.

\hypertarget{web-of-science}{%
\subsubsection{Web of Science}\label{web-of-science}}

\hypertarget{query-1-1}{%
\paragraph{Query 1:}\label{query-1-1}}

WC=(biom* OR health OR allergy OR cell biology OR cardio* OR hematology
OR immunology OR life sciences biomedicine other topics OR medical
informatics OR neuro* OR oncology OR pharmacology OR radiology, nuclear
medicine \& medical imaging OR research \& experimental medicine OR
substance abuse OR optics) AND AK=(longitudinal study OR repeated
measures study) NOT ALL=(review OR meta analysis) NOT AK=(model* AND
study design) NOT KP=(model)

Hits: 4,716

Comments: \textcolor{blue}{This query seems to be good}.

Web of Science allows to specify more fields that result in a more
targeted search. The last two parts of the query (AK and KP) removed
studies method or tutorial papers from journals such as \emph{Statistics
in Medicine}.

\hypertarget{for-methods-on-longitudinal-data}{%
\subsection{For Methods on Longitudinal
Data}\label{for-methods-on-longitudinal-data}}

\hypertarget{web-of-science-1}{%
\subsubsection{Web of Science}\label{web-of-science-1}}

\hypertarget{query-1-2}{%
\paragraph{Query 1:}\label{query-1-2}}

AK=((longitudinal OR repeated measures OR longitudinal data) AND (model
OR design)) NOT ALL=(review OR meta analysis) NOT ALL=(survival
analysis)

Hits: 3,071

Comments: \textcolor{blue}{This query seems to be good}.

This query returns papers that deal with methods for longitudinal
analysis. Two additional options can be selected: 1) include only
articles (which reduces the number of hits to 2,936 as book chapters and
editorials are omitted) and 2) select from the 01/01/2000 until today
(which could be reasonable as the increment of models has occurred
during the last two decades. This option reduces the number to papers to
2,849).

\hypertarget{criteria}{%
\section{Criteria}\label{criteria}}

\hypertarget{inclusion-criteria}{%
\subsection{Inclusion Criteria}\label{inclusion-criteria}}

\begin{itemize}
\item
  methods paper see new methods developed
\item
  application
\end{itemize}

\hypertarget{exclusion-criteria}{%
\subsection{Exclusion Criteria}\label{exclusion-criteria}}

\hypertarget{additional-resources}{%
\section{Additional Resources}\label{additional-resources}}

\hypertarget{comparison}{%
\section{Comparison (?)}\label{comparison}}

\hypertarget{data-extraction}{%
\section{Data Extraction}\label{data-extraction}}

\hypertarget{data-synthesis-strategy}{%
\section{Data Synthesis Strategy}\label{data-synthesis-strategy}}

\hypertarget{references}{%
\section{References}\label{references}}

\hypertarget{refs}{}
\begin{CSLReferences}{0}{0}
\leavevmode\vadjust pre{\hypertarget{ref-edwards2000}{}}%
\CSLLeftMargin{1. }%
\CSLRightInline{Edwards LJ. Modern statistical techniques for the
analysis of longitudinal data in biomedical research. \emph{Pediatric
Pulmonology}. 2000;30(4):330-344.
doi:\url{https://doi.org/10.1002/1099-0496(200010)30:4\%3C330::AID-PPUL10\%3E3.0.CO;2-D}}

\leavevmode\vadjust pre{\hypertarget{ref-zeger1992}{}}%
\CSLLeftMargin{2. }%
\CSLRightInline{Zeger SL, Liang K-Y. An overview of methods for the
analysis of longitudinal data. \emph{Statistics in Medicine}.
1992;11(14-15):1825-1839.
doi:\url{https://doi.org/10.1002/sim.4780111406}}

\leavevmode\vadjust pre{\hypertarget{ref-caruana2015}{}}%
\CSLLeftMargin{3. }%
\CSLRightInline{Caruana EJ, Roman M, Hernández-Sánchez J, Solli P.
Longitudinal studies. \emph{Journal of Thoracic Disease}.
2015;7(11):E537-40.}

\leavevmode\vadjust pre{\hypertarget{ref-mundo2022a}{}}%
\CSLLeftMargin{4. }%
\CSLRightInline{Mundo AI, Tipton JR, Muldoon TJ. Generalized additive
models to analyze nonlinear trends in biomedical longitudinal data using
r: Beyond repeated measures {ANOVA} and linear mixed models.
\emph{Statistics in Medicine}. Published online July 2022.}

\leavevmode\vadjust pre{\hypertarget{ref-thiese2015}{}}%
\CSLLeftMargin{5. }%
\CSLRightInline{Thiese MS, Arnold ZC, Walker SD. The misuse and abuse of
statistics in biomedical research. \emph{Biochem Med (Zagreb)}.
2015;25(1):5-11.}

\leavevmode\vadjust pre{\hypertarget{ref-liu2010}{}}%
\CSLLeftMargin{6. }%
\CSLRightInline{Liu C, Cripe TP, Kim M-O. Statistical issues in
longitudinal data analysis for treatment efficacy studies in the
biomedical sciences. \emph{Molecular Therapy}. 2010;18(9):1724-1730.
doi:\url{https://doi.org/10.1038/mt.2010.127}}

\leavevmode\vadjust pre{\hypertarget{ref-pinheiro2000}{}}%
\CSLLeftMargin{7. }%
\CSLRightInline{Linear mixed-effects models: Basic concepts and
examples. In: \emph{Mixed-Effects Models in s and s-PLUS}. Springer New
York; 2000:3-56.
doi:\href{https://doi.org/10.1007/0-387-22747-4_1}{10.1007/0-387-22747-4\_1}}

\leavevmode\vadjust pre{\hypertarget{ref-jiang2021}{}}%
\CSLLeftMargin{8. }%
\CSLRightInline{Jiang J, Nguyen T. \emph{Linear and Generalized Linear
Mixed Models and Their Applications}. 2nd ed. Springer; 2021.}

\leavevmode\vadjust pre{\hypertarget{ref-hastie2017}{}}%
\CSLLeftMargin{9. }%
\CSLRightInline{Hastie TJ. \emph{Statistical Models in {S}}. (Chambers
JM, Hastie TJ, eds.). Routledge; 2017.}

\leavevmode\vadjust pre{\hypertarget{ref-rosa2004}{}}%
\CSLLeftMargin{10. }%
\CSLRightInline{Rosa GJM, Gianola D, Padovani CR. Bayesian longitudinal
data analysis with mixed models and thick-tailed distributions using
{MCMC}. \emph{Journal of Applied Statistics}. 2004;31(7):855-873.}

\leavevmode\vadjust pre{\hypertarget{ref-ballinger2004}{}}%
\CSLLeftMargin{11. }%
\CSLRightInline{Ballinger GA. Using generalized estimating equations for
longitudinal data analysis. \emph{Organizational Research Methods}.
2004;7(2):127-150.}

\leavevmode\vadjust pre{\hypertarget{ref-wang2014}{}}%
\CSLLeftMargin{12. }%
\CSLRightInline{Wang M. Generalized estimating equations in longitudinal
data analysis: A review and recent developments. \emph{Advances in
Statistics}. 2014;2014:1-11.}

\leavevmode\vadjust pre{\hypertarget{ref-tian2020}{}}%
\CSLLeftMargin{13. }%
\CSLRightInline{Tian Q, Qin L, Zhu W, Xiong S, Wu B. Analysis of factors
contributing to postoperative body weight change in patients with
gastric cancer: Based on generalized estimation equation. \emph{PeerJ}.
2020;8(e9390):e9390.}

\leavevmode\vadjust pre{\hypertarget{ref-sevik2017}{}}%
\CSLLeftMargin{14. }%
\CSLRightInline{Şevik M, Doğan M. Epidemiological and molecular studies
on lumpy skin disease outbreaks in turkey during 2014-2015.
\emph{Transboundary and Emerging Diseases}. 2017;64(4):1268-1279.}

\leavevmode\vadjust pre{\hypertarget{ref-gueorguieva2004}{}}%
\CSLLeftMargin{15. }%
\CSLRightInline{Gueorguieva R, Krystal JH. {Move Over ANOVA: Progress in
Analyzing Repeated-Measures Data andIts Reflection in Papers Published
in the Archives of General Psychiatry}. \emph{Archives of General
Psychiatry}. 2004;61(3):310-317.
doi:\href{https://doi.org/10.1001/archpsyc.61.3.310}{10.1001/archpsyc.61.3.310}}

\leavevmode\vadjust pre{\hypertarget{ref-mccullagh2019}{}}%
\CSLLeftMargin{16. }%
\CSLRightInline{McCullagh P, Nelder JA. \emph{Generalized Linear
Models}. Routledge; 2019.}

\leavevmode\vadjust pre{\hypertarget{ref-breslow1993}{}}%
\CSLLeftMargin{17. }%
\CSLRightInline{Breslow NE, Clayton DG. Approximate inference in
generalized linear mixed models. \emph{Journal of the American
Statistical Association}. 1993;88(421):9-25.
doi:\href{https://doi.org/10.1080/01621459.1993.10594284}{10.1080/01621459.1993.10594284}}

\leavevmode\vadjust pre{\hypertarget{ref-jarvis2016}{}}%
\CSLLeftMargin{18. }%
\CSLRightInline{Jarvis MF, Williams M. Irreproducibility in preclinical
biomedical research: Perceptions, uncertainties, and knowledge gaps.
\emph{Trends in Pharmacological Sciences}. 2016;37(4):290-302.
doi:\url{https://doi.org/10.1016/j.tips.2015.12.001}}

\leavevmode\vadjust pre{\hypertarget{ref-turkiewicz2018}{}}%
\CSLLeftMargin{19. }%
\CSLRightInline{Turkiewicz A, Luta G, Hughes HV, Ranstam J. Statistical
mistakes and how to avoid them {\textendash} lessons learned from the
reproducibility crisis. \emph{Osteoarthritis and Cartilage}.
2018;26(11):1409-1411.
doi:\href{https://doi.org/10.1016/j.joca.2018.07.017}{10.1016/j.joca.2018.07.017}}

\leavevmode\vadjust pre{\hypertarget{ref-gosselin2020}{}}%
\CSLLeftMargin{20. }%
\CSLRightInline{Gosselin R-D. Statistical analysis must improve to
address the reproducibility crisis: The {ACcess} to transparent
statistics ({ACTS}) call to action. \emph{Bioessays}.
2020;42(1):e1900189.}

\leavevmode\vadjust pre{\hypertarget{ref-lang2015}{}}%
\CSLLeftMargin{21. }%
\CSLRightInline{Lang TA, Altman DG. Basic statistical reporting for
articles published in biomedical journals: The {``statistical analyses
and methods in the published literature''} or the {SAMPL} guidelines.
\emph{Int J Nurs Stud}. 2015;52(1):5-9.}

\leavevmode\vadjust pre{\hypertarget{ref-gentleman2007}{}}%
\CSLLeftMargin{22. }%
\CSLRightInline{Gentleman R, Lang DT. Statistical analyses and
reproducible research. \emph{Journal of Computational and Graphical
Statistics}. 2007;16(1):1-23. Accessed August 16, 2022.
\url{http://www.jstor.org/stable/27594227}}

\leavevmode\vadjust pre{\hypertarget{ref-indrayan2020}{}}%
\CSLLeftMargin{23. }%
\CSLRightInline{Indrayan A. Reporting of basic statistical methods in
biomedical journals: Improved {SAMPL} guidelines. \emph{Indian
Pediatrics}. 2020;57(1):43-48.
doi:\href{https://doi.org/10.1007/s13312-020-1702-4}{10.1007/s13312-020-1702-4}}

\end{CSLReferences}



\end{document}
