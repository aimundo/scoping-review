% Options for packages loaded elsewhere
\PassOptionsToPackage{unicode}{hyperref}
\PassOptionsToPackage{hyphens}{url}
\PassOptionsToPackage{dvipsnames,svgnames,x11names}{xcolor}
%
\documentclass[
]{article}

\usepackage{amsmath,amssymb}
\usepackage{lmodern}
\usepackage{iftex}
\ifPDFTeX
  \usepackage[T1]{fontenc}
  \usepackage[utf8]{inputenc}
  \usepackage{textcomp} % provide euro and other symbols
\else % if luatex or xetex
  \usepackage{unicode-math}
  \defaultfontfeatures{Scale=MatchLowercase}
  \defaultfontfeatures[\rmfamily]{Ligatures=TeX,Scale=1}
\fi
% Use upquote if available, for straight quotes in verbatim environments
\IfFileExists{upquote.sty}{\usepackage{upquote}}{}
\IfFileExists{microtype.sty}{% use microtype if available
  \usepackage[]{microtype}
  \UseMicrotypeSet[protrusion]{basicmath} % disable protrusion for tt fonts
}{}
\makeatletter
\@ifundefined{KOMAClassName}{% if non-KOMA class
  \IfFileExists{parskip.sty}{%
    \usepackage{parskip}
  }{% else
    \setlength{\parindent}{0pt}
    \setlength{\parskip}{6pt plus 2pt minus 1pt}}
}{% if KOMA class
  \KOMAoptions{parskip=half}}
\makeatother
\usepackage{xcolor}
\usepackage[top=20mm,left=20mm,heightrounded]{geometry}
\setlength{\emergencystretch}{3em} % prevent overfull lines
\setcounter{secnumdepth}{5}
% Make \paragraph and \subparagraph free-standing
\ifx\paragraph\undefined\else
  \let\oldparagraph\paragraph
  \renewcommand{\paragraph}[1]{\oldparagraph{#1}\mbox{}}
\fi
\ifx\subparagraph\undefined\else
  \let\oldsubparagraph\subparagraph
  \renewcommand{\subparagraph}[1]{\oldsubparagraph{#1}\mbox{}}
\fi


\providecommand{\tightlist}{%
  \setlength{\itemsep}{0pt}\setlength{\parskip}{0pt}}\usepackage{longtable,booktabs,array}
\usepackage{calc} % for calculating minipage widths
% Correct order of tables after \paragraph or \subparagraph
\usepackage{etoolbox}
\makeatletter
\patchcmd\longtable{\par}{\if@noskipsec\mbox{}\fi\par}{}{}
\makeatother
% Allow footnotes in longtable head/foot
\IfFileExists{footnotehyper.sty}{\usepackage{footnotehyper}}{\usepackage{footnote}}
\makesavenoteenv{longtable}
\usepackage{graphicx}
\makeatletter
\def\maxwidth{\ifdim\Gin@nat@width>\linewidth\linewidth\else\Gin@nat@width\fi}
\def\maxheight{\ifdim\Gin@nat@height>\textheight\textheight\else\Gin@nat@height\fi}
\makeatother
% Scale images if necessary, so that they will not overflow the page
% margins by default, and it is still possible to overwrite the defaults
% using explicit options in \includegraphics[width, height, ...]{}
\setkeys{Gin}{width=\maxwidth,height=\maxheight,keepaspectratio}
% Set default figure placement to htbp
\makeatletter
\def\fps@figure{htbp}
\makeatother
\newlength{\cslhangindent}
\setlength{\cslhangindent}{1.5em}
\newlength{\csllabelwidth}
\setlength{\csllabelwidth}{3em}
\newlength{\cslentryspacingunit} % times entry-spacing
\setlength{\cslentryspacingunit}{\parskip}
\newenvironment{CSLReferences}[2] % #1 hanging-ident, #2 entry spacing
 {% don't indent paragraphs
  \setlength{\parindent}{0pt}
  % turn on hanging indent if param 1 is 1
  \ifodd #1
  \let\oldpar\par
  \def\par{\hangindent=\cslhangindent\oldpar}
  \fi
  % set entry spacing
  \setlength{\parskip}{#2\cslentryspacingunit}
 }%
 {}
\usepackage{calc}
\newcommand{\CSLBlock}[1]{#1\hfill\break}
\newcommand{\CSLLeftMargin}[1]{\parbox[t]{\csllabelwidth}{#1}}
\newcommand{\CSLRightInline}[1]{\parbox[t]{\linewidth - \csllabelwidth}{#1}\break}
\newcommand{\CSLIndent}[1]{\hspace{\cslhangindent}#1}

\usepackage{lineno}
\usepackage{setspace}
\linenumbers
\doublespacing

%Commands for Comments from BN

\newcommand{\BN}[1]{\textcolor{red}{#1}}
\usepackage{ulem}
%\pdfstringdefDisableCommands{\let\sout\relax}
\makeatletter
\makeatother
\makeatletter
\makeatother
\makeatletter
\@ifpackageloaded{caption}{}{\usepackage{caption}}
\AtBeginDocument{%
\ifdefined\contentsname
  \renewcommand*\contentsname{Table of contents}
\else
  \newcommand\contentsname{Table of contents}
\fi
\ifdefined\listfigurename
  \renewcommand*\listfigurename{List of Figures}
\else
  \newcommand\listfigurename{List of Figures}
\fi
\ifdefined\listtablename
  \renewcommand*\listtablename{List of Tables}
\else
  \newcommand\listtablename{List of Tables}
\fi
\ifdefined\figurename
  \renewcommand*\figurename{Figure}
\else
  \newcommand\figurename{Figure}
\fi
\ifdefined\tablename
  \renewcommand*\tablename{Table}
\else
  \newcommand\tablename{Table}
\fi
}
\@ifpackageloaded{float}{}{\usepackage{float}}
\floatstyle{ruled}
\@ifundefined{c@chapter}{\newfloat{codelisting}{h}{lop}}{\newfloat{codelisting}{h}{lop}[chapter]}
\floatname{codelisting}{Listing}
\newcommand*\listoflistings{\listof{codelisting}{List of Listings}}
\makeatother
\makeatletter
\@ifpackageloaded{caption}{}{\usepackage{caption}}
\@ifpackageloaded{subcaption}{}{\usepackage{subcaption}}
\makeatother
\makeatletter
\@ifpackageloaded{tcolorbox}{}{\usepackage[many]{tcolorbox}}
\makeatother
\makeatletter
\@ifundefined{shadecolor}{\definecolor{shadecolor}{rgb}{.97, .97, .97}}
\makeatother
\makeatletter
\makeatother
\ifLuaTeX
  \usepackage{selnolig}  % disable illegal ligatures
\fi
\IfFileExists{bookmark.sty}{\usepackage{bookmark}}{\usepackage{hyperref}}
\IfFileExists{xurl.sty}{\usepackage{xurl}}{} % add URL line breaks if available
\urlstyle{same} % disable monospaced font for URLs
\hypersetup{
  pdftitle={Scoping Review Protocol: Statistical Models for Longitudinal Data in Health and Biomedical Research: Current State, Challenges, and Opportunities},
  pdfauthor={Ariel I. Mundo Ortiz},
  colorlinks=true,
  linkcolor={blue},
  filecolor={Maroon},
  citecolor={Blue},
  urlcolor={Blue},
  pdfcreator={LaTeX via pandoc}}

\title{Scoping Review Protocol: Statistical Models for Longitudinal Data
in Health and Biomedical Research: Current State, Challenges, and
Opportunities}
\author{Ariel I. Mundo Ortiz}
\date{2022-09-07}

\begin{document}
\maketitle
\ifdefined\Shaded\renewenvironment{Shaded}{\begin{tcolorbox}[borderline west={3pt}{0pt}{shadecolor}, frame hidden, interior hidden, enhanced, breakable, sharp corners, boxrule=0pt]}{\end{tcolorbox}}\fi

\renewcommand*\contentsname{Table of contents}
{
\hypersetup{linkcolor=}
\setcounter{tocdepth}{3}
\tableofcontents
}
\hypertarget{notes}{%
\section{Notes}\label{notes}}

\textbf{As of Sept 7, 2022 this document follows the structure
recommended by PRISMA-P}
https://prisma-statement.org/documents/PRISMA-P-checklist.pdf

\hypertarget{registration}{%
\section{Registration}\label{registration}}

This section will be populated with the registration number and registry
name once the protocol is submitted for peer review.

\hypertarget{author-contributions}{%
\section{Author Contributions}\label{author-contributions}}

\begin{itemize}
\tightlist
\item
  AM: Writing, query design, data extraction and analysis . . .
\end{itemize}

Other authors to add later

\hypertarget{amendements}{%
\section{Amendements}\label{amendements}}

Protocol amendments resulting from peer review will be indicated in this
section indicating the date of each amendment.

\hypertarget{support}{%
\section{Support}\label{support}}

This section will indicate the sources of financial or other support for
the review

\hypertarget{sources}{%
\subsection{Sources}\label{sources}}

\hypertarget{introduction}{%
\section{Introduction}\label{introduction}}

\hypertarget{rationale}{%
\subsection{Rationale}\label{rationale}}

Longitudinal studies are frequently used in the health sciences
(biomedical research, epidemiology, public health, among others) as they
allow to examine how the temporal effect of a treatment or an
intervention, in contrast to a cross-sectional study, which only allows
to examine the effect of the intervention at a single time point. When
compared their cross-sectional counterparts, longitudinal studies allow
for increased statistical power and more cost efficient
strategies\textsuperscript{1,2}. However, the statistical analysis of
longitudinal data requires to take into consideration factors such as
data missingness, correlation, and non-linear trends, which do not occur
on cross-sectional data\textsuperscript{3,4}. In other words, there is
an ``analytic cost'' associated with the increased complexity of
longitudinal data\textsuperscript{2}.

This additional layer of complexity has led to a problem of model
misspecification in the statistical analysis of the data (i.e., the use
of a statistical model that is not coherent with the data), which has
been reported to occur in many fields, including the health
sciences\textsuperscript{5}. For example, in a landmark study Liu et
al.~showed that in a subset of papers in the biomedical sciences, the
most popular model used to analyze longitudinal data was the analysis of
variance (ANOVA, an approach that fails to take into account the
correlation between measures over time), and that only 18\% of the
studies analyzed used models intended for longitudinal analysis while
checking that the assumptions of the model were satisfied by the
data\textsuperscript{6}.

Historically, the repeated measures ANOVA (rm-ANOVA, a statistical model
for longitudinal data) has been the preferred method in the health
sciences to analyze longitudinal data, despite the fact that the
multiple assumptions required by this model are frequently not satisfied
by the data collected in longitudinal studies\textsuperscript{4}. On the
other hand, the last 30 years have seen incredible progress in the field
of Statistics with the development of statistical models for
longitudinal data that relax the assumptions of rm-ANOVA. Linear mixed
models, generalized additive mixed models, and generalized estimating
equations are among these modern statistical models developed for
longitudinal data\textsuperscript{7--11}. From these statistical
methods, linear mixed models and generalized estimating equations are
the two classes of models that have been frequently applied to analyze
longitudinal data in the health sciences during the last
decade\textsuperscript{12--14}.

However, modern statistical methods that are suited to analyze
longitudinal data have been the exception rather than the norm in the
health sciences. In 2001, a study reported that only 30\% of the
clinical trials analyzed used linear mixed models to analyze their
results, and that the preferred method of analysis continued to be
rm-ANOVA\textsuperscript{15} (in comparison, McCullagh and Nelder's
seminal book on the generalized linear model (GLM) was published in
1989\textsuperscript{16}, and there was ongoing work on the extension of
the GLM framework to the mixed model case by 1993\textsuperscript{17}).
Apart from the aforementioned study, there are not recent papers that
examine the use of modern statistical methods for longitudinal data in
the health sciences. Such information is critical to understand if the
use of these methods has increased or decreased in the field over the
last 20 years, and the reasons behind such changes.

Additionally, the reproducibility crisis is an ongoing issue in the
health sciences\textsuperscript{18,19}, a major component of it being
the misuse and lack of reproducibility of statistical
analyses\textsuperscript{20,21}. Despite the fact that the landscape of
statistical software has vastly increased in the last decade with many
statistical computational tools now available to researchers,
reproducibility standards vary between each computational
tool\textsuperscript{22}. Furthermore, there is still high variability
in the amount of statistical reporting across
journals\textsuperscript{23}. Understanding what statistical
computational tools are used nowadays by researchers in the health
sciences can provide an assessment of the advances in the field towards
research reproducibility, while identifying limitations that might still
be in place.

\hypertarget{objectives}{%
\section{Objectives}\label{objectives}}

This study aims to:

\begin{itemize}
\item
  Identify the different statistical models for longitudinal data that
  are used in the health sciences in order to measure the current extent
  in the adoption of modern statistical methods by the field (Aim 1a)
\item
  Summarize the computational tools used by researchers in the health
  sciences to statistically analyze longitudinal data to understand the
  current status of the field with regards to reproducibility. (Aim 1b)
\item
  List statistical methods for longitudinal data developed within the
  last decade in order to showcase newer methods that may be applicable
  for longitudinal data in a biomedical/health context. (Aim 2)
\end{itemize}

\hypertarget{review-question}{%
\section{Review Question}\label{review-question}}

\begin{itemize}
\item
  What are the statistical methods used in biomedical/health sciences
  research?
\item
  Has the use of modern statistical methods increased in the field
  during the last 20 years?
\item
  What computational tools are most commonly used by researchers to
  analyze longitudinal data, and how in turn this affects
  reproducibility?
\item
  What are most recent statistical methods developed for longitudinal
  data, and how can they be applied in the health sciences?
\end{itemize}

\hypertarget{methods}{%
\section{Methods}\label{methods}}

\hypertarget{types-of-studies}{%
\subsection{Types of Studies}\label{types-of-studies}}

For all the study aims, studies included in the analysis correspond to
peer-reviewed publications in English.

\hypertarget{eligibility-criteria}{%
\subsection{Eligibility Criteria}\label{eligibility-criteria}}

\hypertarget{for-the-application-of-modern-statistical-models-on-longitudinal-biomedicalhealth-data-aims-1a-and-1b}{%
\subsubsection{For the Application of Modern Statistical Models on
Longitudinal Biomedical/Health Data (Aims 1a and
1b)}\label{for-the-application-of-modern-statistical-models-on-longitudinal-biomedicalhealth-data-aims-1a-and-1b}}

\hypertarget{inclusion-criteria}{%
\paragraph{Inclusion Criteria}\label{inclusion-criteria}}

\begin{itemize}
\item
  Articles that:

  \begin{itemize}
  \item
    Are written in English
  \item
    Belong to the biomedical/health sciences fields
  \item
    Describe the collection and analysis of continous or discrete
    longitudinal data
  \item
    Indicate the statistical model used to analyze the data
  \item
    Report the results of their statistical analyses
  \end{itemize}
\end{itemize}

\hypertarget{exclusion-criteria}{%
\paragraph{Exclusion Criteria}\label{exclusion-criteria}}

\begin{itemize}
\item
  Cross-sectional studies
\item
  Tutorials that present the application of existing statistical methods
  to biomedical/health data
\item
  Reviews, meta-analyses, or systematic reviews on existing statistical
  methods for longitudinal data
\item
  Studies that use only descriptive statistics to summarize/analyze the
  data
\item
  Studies that collect and analyze categorical data
\end{itemize}

\hypertarget{for-methods-on-longitudinal-data-aim-2}{%
\subsubsection{For Methods on Longitudinal Data (Aim
2)}\label{for-methods-on-longitudinal-data-aim-2}}

\hypertarget{inclusion-criteria-1}{%
\paragraph{Inclusion Criteria}\label{inclusion-criteria-1}}

\begin{itemize}
\item
  Articles that:

  \begin{itemize}
  \item
    Are written in English
  \item
    Present new methodologies or significant improvements to existing
    methods for longitudinal data
  \end{itemize}
\end{itemize}

\hypertarget{exclusion-criteria-1}{%
\paragraph{Exclusion Criteria}\label{exclusion-criteria-1}}

\begin{itemize}
\item
  Systematic reviews, meta-analyses, or reviews of statistical methods
  for longitudinal data
\item
  Tutorials that present the application of existing statistical methods
  to biomedical/health longitudinal data
\end{itemize}

\hypertarget{information-sources}{%
\subsection{Information Sources}\label{information-sources}}

Studies will be retrieved from PubMed and Web of Science.

\hypertarget{search-strategy}{%
\subsection{Search Strategy}\label{search-strategy}}

\hypertarget{for-the-application-of-modern-models-on-longitudinal-biomedicalhealth-data}{%
\subsubsection{For the Application of Modern Models on Longitudinal
Biomedical/Health
Data}\label{for-the-application-of-modern-models-on-longitudinal-biomedicalhealth-data}}

\hypertarget{pubmed}{%
\paragraph{PubMed}\label{pubmed}}

Note: I removed the query I had with specific names of specific
statistical models (as the results were too broad and non-specific).

\hypertarget{query-2}{%
\subparagraph{Query 2:}\label{query-2}}

(biomedical OR health) AND ((repeated measures) OR (longitudinal study)
\BN{OR longitudinal data}) AND ((statistical analyses) OR (statistical
analysis)) NOT \BN{\sout{((review) OR
(meta analysis))}}
\BN{when you put NOT that might exclude papers with Review and meta analysis as words in the paper}

Hits: 12,617

Response to comment: I followed your advice and re-wrote the query, but
now I was sure to exclude meta-analysis and review papers by \emph{type
of publication}, and papers that are classified in PubMed as devoted to
Statistical methodologies (not about application of methods to
longitudinal data):

\hypertarget{query-3-modified-query-2}{%
\subparagraph{Query 3 (modified Query
2):}\label{query-3-modified-query-2}}

biomedical OR health) AND ((repeated measures) OR (longitudinal study)
OR longitudinal data) AND ((statistical analyses) OR (statistical
analysis)) NOT (Review{[}Publication Type{]} OR Meta
analy*{[}Publication Type{]}) NOT ( ``Statistics as
Topic/methods''{[}Majr{]} OR ``Statistics as Topic/statistics and
numerical data''{[}Majr{]} OR ``Models, Statistical''{[}Mesh{]} OR
``Research Design''{[}Mesh{]})

Hits: 10,948

Comments: \textcolor{blue}{This query is better than Query 2}.

Papers from this query appear to be good. The query catches many papers
from psychology and psychiatry, but the ones I checked did said used
linear mixed models or regression in their analyses.A few of them still
deal with methodologies, but seems to be much more less than in the
previous query.

\hypertarget{web-of-science}{%
\paragraph{Web of Science}\label{web-of-science}}

\hypertarget{query-1}{%
\subparagraph{Query 1:}\label{query-1}}

WC=(biom* OR health OR allergy OR cell biology OR cardio* OR hematology
OR immunology OR life sciences biomedicine other topics OR medical
informatics OR neuro* OR oncology OR pharmacology OR radiology, nuclear
medicine \& medical imaging OR research \& experimental medicine OR
substance abuse OR optics) AND AK=(longitudinal study OR repeated
measures study) NOT ALL=(review OR meta analysis) NOT AK=(model* AND
study design) NOT KP=(model)

Hits: 4,716

\BN{when you put NOT that might exclude papers with Review and meta analysis as words in the paper}

\BN{Remove the NOT and check the suggestions for the PubMed section}

\hypertarget{query-2-updated-query-1}{%
\subparagraph{Query 2 (Updated Query
1):}\label{query-2-updated-query-1}}

WC=(biom* OR health OR allergy OR cell biology OR cardio* OR hematology
OR immunology OR life sciences biomedicine other topics OR medical
informatics OR neuro* OR oncology OR pharmacology OR radiology, nuclear
medicine \& medical imaging OR research \& experimental medicine OR
substance abuse OR optics) AND AK=(longitudinal study OR repeated
measures study) NOT AK=(model* AND study design) NOT KP=(model) NOT
TI=(review OR meta analy*)

Comments: \textcolor{blue}{Updated this query based on your comments}.

Hits: 4,595

I updated this query based on your comments. Now I used NOT but focused
on the title (hence, TI) so articles with ``review'' or ``meta
analysis'' are excluded. An additional filter is that Web of Science
allows to select the type of publications so I chose ``Article'', and
this would remove review papers, editorials or other materials that are
not research papers.

\hypertarget{for-methods-on-longitudinal-data}{%
\subsubsection{For Methods on Longitudinal
Data}\label{for-methods-on-longitudinal-data}}

\BN{you need to describe this part in your objectives}

\textbf{I re-wrote the query here based on your comments.} After reading
your comments and the queries I created, I realized that the filtering
was not correct. I wrote a new query that I believe better represents
the terms we want to look at.

\hypertarget{query-1-1}{%
\subparagraph{Query 1:}\label{query-1-1}}

(``Models, Statistical'' {[}Mesh{]} OR
``Biostatistics/methods''{[}Mesh{]}) AND (``Longitudinal
Studies''{[}Mesh{]}) NOT (Review{[}Publication Type{]} OR Meta
Analys*{[}Publication Type{]} OR ``editorial''{[}Publication Type{]})
NOT (``survival''{[}Title/abstract{]}) NOT
(``tutorial''{[}title/abstract{]} OR
``orientation''{[}title/abstract{]}) NOT (Humans{[}Mesh{]} OR Adolescent
{[}Mesh{]} OR Animals{[}Mesh{]})

Hits: 142

Comments:

The rationale for this query is to find papers that have been labeled as
dealing with models in Biostatistics or Statistics, that deal with
longitudinal data, but excluding reviews, editorials, meta analysis,
tutorials (that show how to implement an existing model, but not the
development of a new model). Additionally, I added the ``humans'',
``adolescent'', and ``animal'' labels to exclude, because there are
\textbf{many} papers that have all the previous labels but that are
devoted to comparing methods, or about studies with animal or clinical
data (without those last filters for humans, adolescents, and animals
the hits are 14,702).

Again, papers that describe the development of new methods for
longitudinal data should be relatively few when compared to papers that
deal with application, and that is why to me the result of the query
(142 hits) makes sense. I did take a look at the papers of this query
and all of them seem to be about models, which is what we want.

\hypertarget{web-of-science-1}{%
\paragraph{Web of Science}\label{web-of-science-1}}

\hypertarget{query-1-2}{%
\subparagraph{Query 1:}\label{query-1-2}}

AK=((longitudinal OR repeated measures
\BN{sout{OR longitudinal data, already included in longitudinal}}) AND
(model OR design OR \BN{method}))
\sout{NOT ALL=(review OR meta analysis) NOT ALL=(survival
analysis)}

Hits: 3,071

Comments: \textcolor{blue}{This query seems to be good}.

This query returns papers that deal with methods for longitudinal
analysis. Two additional options can be selected: 1) include only
articles (which reduces the number of hits to 2,936 as book chapters and
editorials are omitted) and 2) select from the 01/01/2000 until today
(which could be reasonable as the increment of models has occurred
during the last two decades. This option reduces the number to papers to
2,849).

\BN{we will see when the queries is revised what is the best strategy}

\hypertarget{query-2-updated-query-1-1}{%
\subparagraph{Query 2 (updated Query
1):}\label{query-2-updated-query-1-1}}

I updated the query based on your comments. The query is more specific
now, please see below.

AK=((longitudinal OR repeated measures) AND (model OR design OR method))
NOT TI=(review OR meta analysis) NOT AK=(survival analysis) AND
SU=(stat* OR MATH*) NOT AB=(tutorial)

Hits: 1,978

Comments:

This query looks for papers where the authors used the keywords
longitudinal or repeated measures, and model, or design or method;
excludes papers that have ``review'' or ``meta analysis'' in the title,
excludes those that have keywords of ``survival analysis'', and searches
papers where the main subject has been tagged as statistics or math.
Finally, papers that have the word ``tutorial'' in the abstract are
excluded as well.

\hypertarget{data-collection-and-analysis}{%
\subsection{Data Collection and
Analysis}\label{data-collection-and-analysis}}

\hypertarget{selection-process-and-data-management}{%
\subsubsection{Selection Process and Data
Management}\label{selection-process-and-data-management}}

Two reviewers will independently analyze the database search results and
pre-screen articles based on title and abstract content following the
aforementioned inclusion/exclusion criteria. Manuscripts from the
database(s) search will be stored in the Covidence platform, where
duplicated entries will be removed. For articles where pre-screening
inclusion (or exclusion) is unclear based on title and abstract
analysis, full-text review will be used to make a decision following
review by a third independent reviewer. Manuscripts included after title
and abstract pre-screening will be further screening by two reviewers
that will independently examine the full text of each article.

\hypertarget{data-collection-process}{%
\subsubsection{Data Collection Process}\label{data-collection-process}}

Pilot forms (electronic spreadsheets) will be tested using a
representative sample of the studies to be reviewed (\textasciitilde100
studies). Information in the forms will be independently included by
each reviewer. The forms will be updated (if needed), after the pilot
test by consensus between the reviewers.

Information obtained from each study (statistical method used, software,
etc.) will be tabulated independently by the reviewers in an electronic
spreadsheet.

\hypertarget{data-items}{%
\subsection{Data Items}\label{data-items}}

Aims 1a and 1b:

\begin{itemize}
\item
  Statistical method used
\item
  Sub-area of application (oncology, psychology, public health, etc)
\item
  Computational tool used
\item
  Congruence between statistical method used and the data
\item
  Year of publication
\end{itemize}

Aim 2:

\begin{itemize}
\item
  Statistical method reported
\item
  Assumptions of the model
\item
  Computational tools available for its implementation
\item
  Year of publication
\end{itemize}

\hypertarget{risk-of-bias-in-individual-studies}{%
\subsection{Risk of Bias in Individual
Studies}\label{risk-of-bias-in-individual-studies}}

N/A

\hypertarget{data-synthesis}{%
\subsection{Data Synthesis}\label{data-synthesis}}

The data from the results of each included study will be extracted into
electronic spreadsheets. Summary measures for Aims 1a and 1b include
plots (pie, bar, etc) to show the relative use of each statistical
method reported, computational tool, and congruence between statistical
method and the data. Each plot will be segmented by year to show trends
over time.

For Aim 2, a table will be created where statistical method, year of
publication, assumptions of the model, and applicability to health data
is reported.

\hypertarget{meta-biases}{%
\subsection{Meta-Biases}\label{meta-biases}}

N/A

\hypertarget{references}{%
\section{References}\label{references}}

\hypertarget{refs}{}
\begin{CSLReferences}{0}{0}
\leavevmode\vadjust pre{\hypertarget{ref-edwards2000}{}}%
\CSLLeftMargin{1. }%
\CSLRightInline{Edwards LJ. Modern statistical techniques for the
analysis of longitudinal data in biomedical research. \emph{Pediatric
Pulmonology}. 2000;30(4):330-344.
doi:\url{https://doi.org/10.1002/1099-0496(200010)30:4\%3C330::AID-PPUL10\%3E3.0.CO;2-D}}

\leavevmode\vadjust pre{\hypertarget{ref-zeger1992}{}}%
\CSLLeftMargin{2. }%
\CSLRightInline{Zeger SL, Liang K-Y. An overview of methods for the
analysis of longitudinal data. \emph{Statistics in Medicine}.
1992;11(14-15):1825-1839.
doi:\url{https://doi.org/10.1002/sim.4780111406}}

\leavevmode\vadjust pre{\hypertarget{ref-caruana2015}{}}%
\CSLLeftMargin{3. }%
\CSLRightInline{Caruana EJ, Roman M, Hernández-Sánchez J, Solli P.
Longitudinal studies. \emph{Journal of Thoracic Disease}.
2015;7(11):E537-40.}

\leavevmode\vadjust pre{\hypertarget{ref-mundo2022a}{}}%
\CSLLeftMargin{4. }%
\CSLRightInline{Mundo AI, Tipton JR, Muldoon TJ. Generalized additive
models to analyze nonlinear trends in biomedical longitudinal data using
r: Beyond repeated measures {ANOVA} and linear mixed models.
\emph{Statistics in Medicine}. Published online July 2022.}

\leavevmode\vadjust pre{\hypertarget{ref-thiese2015}{}}%
\CSLLeftMargin{5. }%
\CSLRightInline{Thiese MS, Arnold ZC, Walker SD. The misuse and abuse of
statistics in biomedical research. \emph{Biochem Med (Zagreb)}.
2015;25(1):5-11.}

\leavevmode\vadjust pre{\hypertarget{ref-liu2010}{}}%
\CSLLeftMargin{6. }%
\CSLRightInline{Liu C, Cripe TP, Kim M-O. Statistical issues in
longitudinal data analysis for treatment efficacy studies in the
biomedical sciences. \emph{Molecular Therapy}. 2010;18(9):1724-1730.
doi:\url{https://doi.org/10.1038/mt.2010.127}}

\leavevmode\vadjust pre{\hypertarget{ref-pinheiro2000}{}}%
\CSLLeftMargin{7. }%
\CSLRightInline{Linear mixed-effects models: Basic concepts and
examples. In: \emph{Mixed-Effects Models in s and s-PLUS}. Springer New
York; 2000:3-56.
doi:\href{https://doi.org/10.1007/0-387-22747-4_1}{10.1007/0-387-22747-4\_1}}

\leavevmode\vadjust pre{\hypertarget{ref-jiang2021}{}}%
\CSLLeftMargin{8. }%
\CSLRightInline{Jiang J, Nguyen T. \emph{Linear and Generalized Linear
Mixed Models and Their Applications}. 2nd ed. Springer; 2021.}

\leavevmode\vadjust pre{\hypertarget{ref-hastie2017}{}}%
\CSLLeftMargin{9. }%
\CSLRightInline{Hastie TJ. \emph{Statistical Models in {S}}. (Chambers
JM, Hastie TJ, eds.). Routledge; 2017.}

\leavevmode\vadjust pre{\hypertarget{ref-rosa2004}{}}%
\CSLLeftMargin{10. }%
\CSLRightInline{Rosa GJM, Gianola D, Padovani CR. Bayesian longitudinal
data analysis with mixed models and thick-tailed distributions using
{MCMC}. \emph{Journal of Applied Statistics}. 2004;31(7):855-873.}

\leavevmode\vadjust pre{\hypertarget{ref-ballinger2004}{}}%
\CSLLeftMargin{11. }%
\CSLRightInline{Ballinger GA. Using generalized estimating equations for
longitudinal data analysis. \emph{Organizational Research Methods}.
2004;7(2):127-150.}

\leavevmode\vadjust pre{\hypertarget{ref-wang2014}{}}%
\CSLLeftMargin{12. }%
\CSLRightInline{Wang M. Generalized estimating equations in longitudinal
data analysis: A review and recent developments. \emph{Advances in
Statistics}. 2014;2014:1-11.}

\leavevmode\vadjust pre{\hypertarget{ref-tian2020}{}}%
\CSLLeftMargin{13. }%
\CSLRightInline{Tian Q, Qin L, Zhu W, Xiong S, Wu B. Analysis of factors
contributing to postoperative body weight change in patients with
gastric cancer: Based on generalized estimation equation. \emph{PeerJ}.
2020;8(e9390):e9390.}

\leavevmode\vadjust pre{\hypertarget{ref-sevik2017}{}}%
\CSLLeftMargin{14. }%
\CSLRightInline{Şevik M, Doğan M. Epidemiological and molecular studies
on lumpy skin disease outbreaks in turkey during 2014-2015.
\emph{Transboundary and Emerging Diseases}. 2017;64(4):1268-1279.}

\leavevmode\vadjust pre{\hypertarget{ref-gueorguieva2004}{}}%
\CSLLeftMargin{15. }%
\CSLRightInline{Gueorguieva R, Krystal JH. {Move Over ANOVA: Progress in
Analyzing Repeated-Measures Data andIts Reflection in Papers Published
in the Archives of General Psychiatry}. \emph{Archives of General
Psychiatry}. 2004;61(3):310-317.
doi:\href{https://doi.org/10.1001/archpsyc.61.3.310}{10.1001/archpsyc.61.3.310}}

\leavevmode\vadjust pre{\hypertarget{ref-mccullagh2019}{}}%
\CSLLeftMargin{16. }%
\CSLRightInline{McCullagh P, Nelder JA. \emph{Generalized Linear
Models}. Routledge; 2019.}

\leavevmode\vadjust pre{\hypertarget{ref-breslow1993}{}}%
\CSLLeftMargin{17. }%
\CSLRightInline{Breslow NE, Clayton DG. Approximate inference in
generalized linear mixed models. \emph{Journal of the American
Statistical Association}. 1993;88(421):9-25.
doi:\href{https://doi.org/10.1080/01621459.1993.10594284}{10.1080/01621459.1993.10594284}}

\leavevmode\vadjust pre{\hypertarget{ref-jarvis2016}{}}%
\CSLLeftMargin{18. }%
\CSLRightInline{Jarvis MF, Williams M. Irreproducibility in preclinical
biomedical research: Perceptions, uncertainties, and knowledge gaps.
\emph{Trends in Pharmacological Sciences}. 2016;37(4):290-302.
doi:\url{https://doi.org/10.1016/j.tips.2015.12.001}}

\leavevmode\vadjust pre{\hypertarget{ref-turkiewicz2018}{}}%
\CSLLeftMargin{19. }%
\CSLRightInline{Turkiewicz A, Luta G, Hughes HV, Ranstam J. Statistical
mistakes and how to avoid them {\textendash} lessons learned from the
reproducibility crisis. \emph{Osteoarthritis and Cartilage}.
2018;26(11):1409-1411.
doi:\href{https://doi.org/10.1016/j.joca.2018.07.017}{10.1016/j.joca.2018.07.017}}

\leavevmode\vadjust pre{\hypertarget{ref-gosselin2020}{}}%
\CSLLeftMargin{20. }%
\CSLRightInline{Gosselin R-D. Statistical analysis must improve to
address the reproducibility crisis: The {ACcess} to transparent
statistics ({ACTS}) call to action. \emph{Bioessays}.
2020;42(1):e1900189.}

\leavevmode\vadjust pre{\hypertarget{ref-lang2015}{}}%
\CSLLeftMargin{21. }%
\CSLRightInline{Lang TA, Altman DG. Basic statistical reporting for
articles published in biomedical journals: The {``statistical analyses
and methods in the published literature''} or the {SAMPL} guidelines.
\emph{Int J Nurs Stud}. 2015;52(1):5-9.}

\leavevmode\vadjust pre{\hypertarget{ref-gentleman2007}{}}%
\CSLLeftMargin{22. }%
\CSLRightInline{Gentleman R, Lang DT. Statistical analyses and
reproducible research. \emph{Journal of Computational and Graphical
Statistics}. 2007;16(1):1-23. Accessed August 16, 2022.
\url{http://www.jstor.org/stable/27594227}}

\leavevmode\vadjust pre{\hypertarget{ref-indrayan2020}{}}%
\CSLLeftMargin{23. }%
\CSLRightInline{Indrayan A. Reporting of basic statistical methods in
biomedical journals: Improved {SAMPL} guidelines. \emph{Indian
Pediatrics}. 2020;57(1):43-48.
doi:\href{https://doi.org/10.1007/s13312-020-1702-4}{10.1007/s13312-020-1702-4}}

\end{CSLReferences}



\end{document}
