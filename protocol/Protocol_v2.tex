% Options for packages loaded elsewhere
\PassOptionsToPackage{unicode}{hyperref}
\PassOptionsToPackage{hyphens}{url}
\PassOptionsToPackage{dvipsnames,svgnames,x11names}{xcolor}
%
\documentclass[
]{article}

\usepackage{amsmath,amssymb}
\usepackage{lmodern}
\usepackage{iftex}
\ifPDFTeX
  \usepackage[T1]{fontenc}
  \usepackage[utf8]{inputenc}
  \usepackage{textcomp} % provide euro and other symbols
\else % if luatex or xetex
  \usepackage{unicode-math}
  \defaultfontfeatures{Scale=MatchLowercase}
  \defaultfontfeatures[\rmfamily]{Ligatures=TeX,Scale=1}
\fi
% Use upquote if available, for straight quotes in verbatim environments
\IfFileExists{upquote.sty}{\usepackage{upquote}}{}
\IfFileExists{microtype.sty}{% use microtype if available
  \usepackage[]{microtype}
  \UseMicrotypeSet[protrusion]{basicmath} % disable protrusion for tt fonts
}{}
\makeatletter
\@ifundefined{KOMAClassName}{% if non-KOMA class
  \IfFileExists{parskip.sty}{%
    \usepackage{parskip}
  }{% else
    \setlength{\parindent}{0pt}
    \setlength{\parskip}{6pt plus 2pt minus 1pt}}
}{% if KOMA class
  \KOMAoptions{parskip=half}}
\makeatother
\usepackage{xcolor}
\usepackage[top=20mm,left=20mm,heightrounded]{geometry}
\setlength{\emergencystretch}{3em} % prevent overfull lines
\setcounter{secnumdepth}{5}
% Make \paragraph and \subparagraph free-standing
\ifx\paragraph\undefined\else
  \let\oldparagraph\paragraph
  \renewcommand{\paragraph}[1]{\oldparagraph{#1}\mbox{}}
\fi
\ifx\subparagraph\undefined\else
  \let\oldsubparagraph\subparagraph
  \renewcommand{\subparagraph}[1]{\oldsubparagraph{#1}\mbox{}}
\fi


\providecommand{\tightlist}{%
  \setlength{\itemsep}{0pt}\setlength{\parskip}{0pt}}\usepackage{longtable,booktabs,array}
\usepackage{calc} % for calculating minipage widths
% Correct order of tables after \paragraph or \subparagraph
\usepackage{etoolbox}
\makeatletter
\patchcmd\longtable{\par}{\if@noskipsec\mbox{}\fi\par}{}{}
\makeatother
% Allow footnotes in longtable head/foot
\IfFileExists{footnotehyper.sty}{\usepackage{footnotehyper}}{\usepackage{footnote}}
\makesavenoteenv{longtable}
\usepackage{graphicx}
\makeatletter
\def\maxwidth{\ifdim\Gin@nat@width>\linewidth\linewidth\else\Gin@nat@width\fi}
\def\maxheight{\ifdim\Gin@nat@height>\textheight\textheight\else\Gin@nat@height\fi}
\makeatother
% Scale images if necessary, so that they will not overflow the page
% margins by default, and it is still possible to overwrite the defaults
% using explicit options in \includegraphics[width, height, ...]{}
\setkeys{Gin}{width=\maxwidth,height=\maxheight,keepaspectratio}
% Set default figure placement to htbp
\makeatletter
\def\fps@figure{htbp}
\makeatother
\newlength{\cslhangindent}
\setlength{\cslhangindent}{1.5em}
\newlength{\csllabelwidth}
\setlength{\csllabelwidth}{3em}
\newlength{\cslentryspacingunit} % times entry-spacing
\setlength{\cslentryspacingunit}{\parskip}
\newenvironment{CSLReferences}[2] % #1 hanging-ident, #2 entry spacing
 {% don't indent paragraphs
  \setlength{\parindent}{0pt}
  % turn on hanging indent if param 1 is 1
  \ifodd #1
  \let\oldpar\par
  \def\par{\hangindent=\cslhangindent\oldpar}
  \fi
  % set entry spacing
  \setlength{\parskip}{#2\cslentryspacingunit}
 }%
 {}
\usepackage{calc}
\newcommand{\CSLBlock}[1]{#1\hfill\break}
\newcommand{\CSLLeftMargin}[1]{\parbox[t]{\csllabelwidth}{#1}}
\newcommand{\CSLRightInline}[1]{\parbox[t]{\linewidth - \csllabelwidth}{#1}\break}
\newcommand{\CSLIndent}[1]{\hspace{\cslhangindent}#1}

\usepackage{lineno}
\usepackage{setspace}
\linenumbers
\doublespacing
\makeatletter
\makeatother
\makeatletter
\makeatother
\makeatletter
\@ifpackageloaded{caption}{}{\usepackage{caption}}
\AtBeginDocument{%
\ifdefined\contentsname
  \renewcommand*\contentsname{Table of contents}
\else
  \newcommand\contentsname{Table of contents}
\fi
\ifdefined\listfigurename
  \renewcommand*\listfigurename{List of Figures}
\else
  \newcommand\listfigurename{List of Figures}
\fi
\ifdefined\listtablename
  \renewcommand*\listtablename{List of Tables}
\else
  \newcommand\listtablename{List of Tables}
\fi
\ifdefined\figurename
  \renewcommand*\figurename{Figure}
\else
  \newcommand\figurename{Figure}
\fi
\ifdefined\tablename
  \renewcommand*\tablename{Table}
\else
  \newcommand\tablename{Table}
\fi
}
\@ifpackageloaded{float}{}{\usepackage{float}}
\floatstyle{ruled}
\@ifundefined{c@chapter}{\newfloat{codelisting}{h}{lop}}{\newfloat{codelisting}{h}{lop}[chapter]}
\floatname{codelisting}{Listing}
\newcommand*\listoflistings{\listof{codelisting}{List of Listings}}
\makeatother
\makeatletter
\@ifpackageloaded{caption}{}{\usepackage{caption}}
\@ifpackageloaded{subcaption}{}{\usepackage{subcaption}}
\makeatother
\makeatletter
\@ifpackageloaded{tcolorbox}{}{\usepackage[many]{tcolorbox}}
\makeatother
\makeatletter
\@ifundefined{shadecolor}{\definecolor{shadecolor}{rgb}{.97, .97, .97}}
\makeatother
\makeatletter
\makeatother
\ifLuaTeX
  \usepackage{selnolig}  % disable illegal ligatures
\fi
\IfFileExists{bookmark.sty}{\usepackage{bookmark}}{\usepackage{hyperref}}
\IfFileExists{xurl.sty}{\usepackage{xurl}}{} % add URL line breaks if available
\urlstyle{same} % disable monospaced font for URLs
\hypersetup{
  pdftitle={Scoping Review Protocol: Statistical Models for Longitudinal Data},
  pdfauthor={Ariel I. Mundo Ortiz},
  colorlinks=true,
  linkcolor={blue},
  filecolor={Maroon},
  citecolor={Blue},
  urlcolor={Blue},
  pdfcreator={LaTeX via pandoc}}

\title{Scoping Review Protocol: Statistical Models for Longitudinal
Data}
\author{Ariel I. Mundo Ortiz}
\date{2022-08-11}

\begin{document}
\maketitle
\ifdefined\Shaded\renewenvironment{Shaded}{\begin{tcolorbox}[frame hidden, interior hidden, sharp corners, enhanced, borderline west={3pt}{0pt}{shadecolor}, boxrule=0pt, breakable]}{\end{tcolorbox}}\fi

\renewcommand*\contentsname{Table of contents}
{
\hypersetup{linkcolor=}
\setcounter{tocdepth}{3}
\tableofcontents
}
\hypertarget{background}{%
\section{Background}\label{background}}

Longitudinal studies are frequently used in the health sciences
(biomedical research, epidemiology, public health, among others) as they
allow to examine how the temporal effect of a treatment or an
intervention, in contrast to a cross-sectional study, which only allows
to examine the effect of the intervention at a single time point. When
compared to cross-sectional studies, longitudinal studies allow for
increased statistical power and more cost efficient
strategies\textsuperscript{1,2}. However, the statistical analysis of
longitudinal requires to take into consideration factors such as data
missingness, correlation, and non-linear trends, which do not occur on
cross-sectional data\textsuperscript{3,4}.

This additional layer of complexity in the analysis of longitudinal data
has led to a well documented problem of model misspecification (the use
of a statistical model that is not coherent with the data) in the health
sciences\textsuperscript{4}, which can be partly explained by the fact
that researchers have a tendency to use the same statistical analysis,
methods and tests from other papers without having a clear understanding
of the limitations, assumptions, and applicability of the model in each
situation\textsuperscript{5,6}. For example, in a landmark study Liu et
al.~showed that in a subset of papers in the biomedical sciences, the
most popular model used to analyze longitudinal data was ANOVA (an
approach that fails to take into account the correlation between
measures over time), and that only 18\% of studies used models intended
for longitudinal analysis while checking that the assumptions of the
model were satisfied by the data\textsuperscript{7}.

Historically, the repeated measures analysis of variance (rm-ANOVA) has
been the preferred method in the health sciences to analyze longitudinal
data, despite the fact that frequently, the assumptions required for its
use are not satisfied by the data\textsuperscript{4}. On the other hand,
over the last 30 years the field of Statistics has been able to develop
models for longitudinal data that overcome the limitations of rm-ANOVA,
such as linear mixed models, generalized additive models, Bayesian
models, and generalized estimating equations\textsuperscript{8--12}.
However, the adoption of these modern statistical techniques has been
slow, as showcased by Gueorguieva et al., who showed that by 2001, only
30\% of clinical trials reported in the \emph{Archives of General
Psychiatry} used linear mixed models to analyze their results and that
rm-ANOVA continued to be the preferred method of analysis in most
cases\textsuperscript{13}.

During the last decade, the increased availability of computational
tools to analyze longitudinal data has lead to increased adoption of
modern statistical methods to analyze longitudinal data in the health
sciences\textsuperscript{14--17}. Despite this, it is not known how much
the adoption of these modern statistical methods has increased over the
last 20 years, and what are the reasons that may continue to limit the
knowledge and application of these statistical methods by researchers in
the health sciences. Because research reproducibility continues to be at
the center of the debate on biomedical research {[}citation{]}, there is
a need to better understand the current status of statistical practices
in the health sciences in order to implement changes that can lead to a
harmonized used of statistics.

To answer this question, in this study we surveyed the statistical
methods used in papers dealing with longitudinal data in health sciences
over the last 20 years, in order to gain a better understanding of: 1)
the trends in adoption of modern statistical methods, 2) identify the
most frequent pitfalls in statistical analysis, and 3) provide a
rationale for situations where these methods are still not widely
adopted.

\hypertarget{objective}{%
\section{Objective}\label{objective}}

This study aims to summarize the different statistical models for
longitudinal data that are used in the health sciences, identify the
extent of the adoption of modern statistical methods in the field, and
determine if in each case, model assumptions are checked by researchers
to ensure congruency between the data and the model.

\hypertarget{review-question}{%
\section{Review Question}\label{review-question}}

Summarize the statistical methods used to analyze longitudinal data in
the health sciences to identify which methods are most commonly used,
the applicability of such methods in the context of each study, and gaps
that might exist that prevent the adoption of modern statistical methods
that can be better suited to analyze the data. Additionally, identify if
studies check for model assumptions, and how this in turn impacts the
reported results.

\hypertarget{databases}{%
\section{Databases}\label{databases}}

\begin{itemize}
\tightlist
\item
  PubMed
\item
  Web of Science
\end{itemize}

\hypertarget{search-terms}{%
\section{Search Terms}\label{search-terms}}

\hypertarget{criteria}{%
\section{Criteria}\label{criteria}}

\hypertarget{inclusion-criteria}{%
\subsection{Inclusion Criteria}\label{inclusion-criteria}}

\begin{itemize}
\item
  methods paper see new methods developed
\item
  application
\end{itemize}

\hypertarget{exclusion-criteria}{%
\subsection{Exclusion Criteria}\label{exclusion-criteria}}

\hypertarget{additional-resources}{%
\section{Additional Resources}\label{additional-resources}}

\hypertarget{comparison}{%
\section{Comparison (?)}\label{comparison}}

\hypertarget{data-extraction}{%
\section{Data Extraction}\label{data-extraction}}

\hypertarget{data-synthesis-strategy}{%
\section{Data Synthesis Strategy}\label{data-synthesis-strategy}}

\hypertarget{references}{%
\section{References}\label{references}}

\hypertarget{refs}{}
\begin{CSLReferences}{0}{0}
\leavevmode\vadjust pre{\hypertarget{ref-edwards2000}{}}%
\CSLLeftMargin{1. }%
\CSLRightInline{Edwards LJ. Modern statistical techniques for the
analysis of longitudinal data in biomedical research. \emph{Pediatric
Pulmonology}. 2000;30(4):330-344.
doi:\url{https://doi.org/10.1002/1099-0496(200010)30:4\%3C330::AID-PPUL10\%3E3.0.CO;2-D}}

\leavevmode\vadjust pre{\hypertarget{ref-zeger1992}{}}%
\CSLLeftMargin{2. }%
\CSLRightInline{Zeger SL, Liang K-Y. An overview of methods for the
analysis of longitudinal data. \emph{Statistics in Medicine}.
1992;11(14-15):1825-1839.
doi:\url{https://doi.org/10.1002/sim.4780111406}}

\leavevmode\vadjust pre{\hypertarget{ref-caruana2015}{}}%
\CSLLeftMargin{3. }%
\CSLRightInline{Caruana EJ, Roman M, Hernández-Sánchez J, Solli P.
Longitudinal studies. \emph{Journal of Thoracic Disease}.
2015;7(11):E537-40.}

\leavevmode\vadjust pre{\hypertarget{ref-mundo2022a}{}}%
\CSLLeftMargin{4. }%
\CSLRightInline{Mundo AI, Tipton JR, Muldoon TJ. Generalized additive
models to analyze nonlinear trends in biomedical longitudinal data using
r: Beyond repeated measures {ANOVA} and linear mixed models.
\emph{Statistics in Medicine}. Published online July 2022.}

\leavevmode\vadjust pre{\hypertarget{ref-ercan2007}{}}%
\CSLLeftMargin{5. }%
\CSLRightInline{Ercan I, Yazici B, Yaning Y, et al. Misusage of
statistics in medical research. \emph{European Journal of General
Medicine}. 2007;4(3):128-134.}

\leavevmode\vadjust pre{\hypertarget{ref-thiese2015}{}}%
\CSLLeftMargin{6. }%
\CSLRightInline{Thiese MS, Arnold ZC, Walker SD. The misuse and abuse of
statistics in biomedical research. \emph{Biochem Med (Zagreb)}.
2015;25(1):5-11.}

\leavevmode\vadjust pre{\hypertarget{ref-liu2010}{}}%
\CSLLeftMargin{7. }%
\CSLRightInline{Liu C, Cripe TP, Kim M-O. Statistical issues in
longitudinal data analysis for treatment efficacy studies in the
biomedical sciences. \emph{Molecular Therapy}. 2010;18(9):1724-1730.
doi:\url{https://doi.org/10.1038/mt.2010.127}}

\leavevmode\vadjust pre{\hypertarget{ref-pinheiro2000}{}}%
\CSLLeftMargin{8. }%
\CSLRightInline{Linear mixed-effects models: Basic concepts and
examples. In: \emph{Mixed-Effects Models in s and s-PLUS}. Springer New
York; 2000:3-56.
doi:\href{https://doi.org/10.1007/0-387-22747-4_1}{10.1007/0-387-22747-4\_1}}

\leavevmode\vadjust pre{\hypertarget{ref-jiang2021}{}}%
\CSLLeftMargin{9. }%
\CSLRightInline{Jiang J, Nguyen T. \emph{Linear and Generalized Linear
Mixed Models and Their Applications}. 2nd ed. Springer; 2021.}

\leavevmode\vadjust pre{\hypertarget{ref-hastie2017}{}}%
\CSLLeftMargin{10. }%
\CSLRightInline{Hastie TJ. \emph{Statistical Models in {S}}. (Chambers
JM, Hastie TJ, eds.). Routledge; 2017.}

\leavevmode\vadjust pre{\hypertarget{ref-rosa2004}{}}%
\CSLLeftMargin{11. }%
\CSLRightInline{Rosa GJM, Gianola D, Padovani CR. Bayesian longitudinal
data analysis with mixed models and thick-tailed distributions using
{MCMC}. \emph{Journal of Applied Statistics}. 2004;31(7):855-873.}

\leavevmode\vadjust pre{\hypertarget{ref-ballinger2004}{}}%
\CSLLeftMargin{12. }%
\CSLRightInline{Ballinger GA. Using generalized estimating equations for
longitudinal data analysis. \emph{Organizational Research Methods}.
2004;7(2):127-150.}

\leavevmode\vadjust pre{\hypertarget{ref-gueorguieva2004}{}}%
\CSLLeftMargin{13. }%
\CSLRightInline{Gueorguieva R, Krystal JH. {Move Over ANOVA: Progress in
Analyzing Repeated-Measures Data andIts Reflection in Papers Published
in the Archives of General Psychiatry}. \emph{Archives of General
Psychiatry}. 2004;61(3):310-317.
doi:\href{https://doi.org/10.1001/archpsyc.61.3.310}{10.1001/archpsyc.61.3.310}}

\leavevmode\vadjust pre{\hypertarget{ref-mundo2022b}{}}%
\CSLLeftMargin{14. }%
\CSLRightInline{Mundo AI, Muhammad A, Balza K, Nelson CE, Muldoon TJ.
Longitudinal examination of perfusion and angiogenesis markers in
primary colorectal tumors shows distinct signatures for metronomic and
maximum-tolerated dose strategies. \emph{Neoplasia}. 2022;32:100825.
doi:\href{https://doi.org/10.1016/j.neo.2022.100825}{10.1016/j.neo.2022.100825}}

\leavevmode\vadjust pre{\hypertarget{ref-wang2014}{}}%
\CSLLeftMargin{15. }%
\CSLRightInline{Wang M. Generalized estimating equations in longitudinal
data analysis: A review and recent developments. \emph{Advances in
Statistics}. 2014;2014:1-11.}

\leavevmode\vadjust pre{\hypertarget{ref-tian2020}{}}%
\CSLLeftMargin{16. }%
\CSLRightInline{Tian Q, Qin L, Zhu W, Xiong S, Wu B. Analysis of factors
contributing to postoperative body weight change in patients with
gastric cancer: Based on generalized estimation equation. \emph{PeerJ}.
2020;8(e9390):e9390.}

\leavevmode\vadjust pre{\hypertarget{ref-sevik2017}{}}%
\CSLLeftMargin{17. }%
\CSLRightInline{Şevik M, Doğan M. Epidemiological and molecular studies
on lumpy skin disease outbreaks in turkey during 2014-2015.
\emph{Transboundary and Emerging Diseases}. 2017;64(4):1268-1279.}

\end{CSLReferences}



\end{document}
