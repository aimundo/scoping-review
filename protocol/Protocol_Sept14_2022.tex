% Options for packages loaded elsewhere
\PassOptionsToPackage{unicode}{hyperref}
\PassOptionsToPackage{hyphens}{url}
\PassOptionsToPackage{dvipsnames,svgnames,x11names}{xcolor}
%
\documentclass[
]{article}

\usepackage{amsmath,amssymb}
\usepackage{lmodern}
\usepackage{iftex}
\ifPDFTeX
  \usepackage[T1]{fontenc}
  \usepackage[utf8]{inputenc}
  \usepackage{textcomp} % provide euro and other symbols
\else % if luatex or xetex
  \usepackage{unicode-math}
  \defaultfontfeatures{Scale=MatchLowercase}
  \defaultfontfeatures[\rmfamily]{Ligatures=TeX,Scale=1}
\fi
% Use upquote if available, for straight quotes in verbatim environments
\IfFileExists{upquote.sty}{\usepackage{upquote}}{}
\IfFileExists{microtype.sty}{% use microtype if available
  \usepackage[]{microtype}
  \UseMicrotypeSet[protrusion]{basicmath} % disable protrusion for tt fonts
}{}
\makeatletter
\@ifundefined{KOMAClassName}{% if non-KOMA class
  \IfFileExists{parskip.sty}{%
    \usepackage{parskip}
  }{% else
    \setlength{\parindent}{0pt}
    \setlength{\parskip}{6pt plus 2pt minus 1pt}}
}{% if KOMA class
  \KOMAoptions{parskip=half}}
\makeatother
\usepackage{xcolor}
\usepackage[top=20mm,left=20mm,heightrounded]{geometry}
\setlength{\emergencystretch}{3em} % prevent overfull lines
\setcounter{secnumdepth}{5}
% Make \paragraph and \subparagraph free-standing
\ifx\paragraph\undefined\else
  \let\oldparagraph\paragraph
  \renewcommand{\paragraph}[1]{\oldparagraph{#1}\mbox{}}
\fi
\ifx\subparagraph\undefined\else
  \let\oldsubparagraph\subparagraph
  \renewcommand{\subparagraph}[1]{\oldsubparagraph{#1}\mbox{}}
\fi


\providecommand{\tightlist}{%
  \setlength{\itemsep}{0pt}\setlength{\parskip}{0pt}}\usepackage{longtable,booktabs,array}
\usepackage{calc} % for calculating minipage widths
% Correct order of tables after \paragraph or \subparagraph
\usepackage{etoolbox}
\makeatletter
\patchcmd\longtable{\par}{\if@noskipsec\mbox{}\fi\par}{}{}
\makeatother
% Allow footnotes in longtable head/foot
\IfFileExists{footnotehyper.sty}{\usepackage{footnotehyper}}{\usepackage{footnote}}
\makesavenoteenv{longtable}
\usepackage{graphicx}
\makeatletter
\def\maxwidth{\ifdim\Gin@nat@width>\linewidth\linewidth\else\Gin@nat@width\fi}
\def\maxheight{\ifdim\Gin@nat@height>\textheight\textheight\else\Gin@nat@height\fi}
\makeatother
% Scale images if necessary, so that they will not overflow the page
% margins by default, and it is still possible to overwrite the defaults
% using explicit options in \includegraphics[width, height, ...]{}
\setkeys{Gin}{width=\maxwidth,height=\maxheight,keepaspectratio}
% Set default figure placement to htbp
\makeatletter
\def\fps@figure{htbp}
\makeatother
\newlength{\cslhangindent}
\setlength{\cslhangindent}{1.5em}
\newlength{\csllabelwidth}
\setlength{\csllabelwidth}{3em}
\newlength{\cslentryspacingunit} % times entry-spacing
\setlength{\cslentryspacingunit}{\parskip}
\newenvironment{CSLReferences}[2] % #1 hanging-ident, #2 entry spacing
 {% don't indent paragraphs
  \setlength{\parindent}{0pt}
  % turn on hanging indent if param 1 is 1
  \ifodd #1
  \let\oldpar\par
  \def\par{\hangindent=\cslhangindent\oldpar}
  \fi
  % set entry spacing
  \setlength{\parskip}{#2\cslentryspacingunit}
 }%
 {}
\usepackage{calc}
\newcommand{\CSLBlock}[1]{#1\hfill\break}
\newcommand{\CSLLeftMargin}[1]{\parbox[t]{\csllabelwidth}{#1}}
\newcommand{\CSLRightInline}[1]{\parbox[t]{\linewidth - \csllabelwidth}{#1}\break}
\newcommand{\CSLIndent}[1]{\hspace{\cslhangindent}#1}

\usepackage{lineno}
\usepackage{setspace}
\linenumbers
\doublespacing

%Commands for Comments from BN

%\newcommand{\BN}[1]{\textcolor{red}{#1}}
%\usepackage{ulem}
%\pdfstringdefDisableCommands{\let\sout\relax}
\usepackage{booktabs}
\usepackage{longtable}
\usepackage{array}
\usepackage{multirow}
\usepackage{wrapfig}
\usepackage{float}
\usepackage{colortbl}
\usepackage{pdflscape}
\usepackage{tabu}
\usepackage{threeparttable}
\usepackage{threeparttablex}
\usepackage[normalem]{ulem}
\usepackage{makecell}
\usepackage{xcolor}
\makeatletter
\makeatother
\makeatletter
\makeatother
\makeatletter
\@ifpackageloaded{caption}{}{\usepackage{caption}}
\AtBeginDocument{%
\ifdefined\contentsname
  \renewcommand*\contentsname{Table of contents}
\else
  \newcommand\contentsname{Table of contents}
\fi
\ifdefined\listfigurename
  \renewcommand*\listfigurename{List of Figures}
\else
  \newcommand\listfigurename{List of Figures}
\fi
\ifdefined\listtablename
  \renewcommand*\listtablename{List of Tables}
\else
  \newcommand\listtablename{List of Tables}
\fi
\ifdefined\figurename
  \renewcommand*\figurename{Figure}
\else
  \newcommand\figurename{Figure}
\fi
\ifdefined\tablename
  \renewcommand*\tablename{Table}
\else
  \newcommand\tablename{Table}
\fi
}
\@ifpackageloaded{float}{}{\usepackage{float}}
\floatstyle{ruled}
\@ifundefined{c@chapter}{\newfloat{codelisting}{h}{lop}}{\newfloat{codelisting}{h}{lop}[chapter]}
\floatname{codelisting}{Listing}
\newcommand*\listoflistings{\listof{codelisting}{List of Listings}}
\makeatother
\makeatletter
\@ifpackageloaded{caption}{}{\usepackage{caption}}
\@ifpackageloaded{subcaption}{}{\usepackage{subcaption}}
\makeatother
\makeatletter
\@ifpackageloaded{tcolorbox}{}{\usepackage[many]{tcolorbox}}
\makeatother
\makeatletter
\@ifundefined{shadecolor}{\definecolor{shadecolor}{rgb}{.97, .97, .97}}
\makeatother
\makeatletter
\makeatother
\ifLuaTeX
  \usepackage{selnolig}  % disable illegal ligatures
\fi
\IfFileExists{bookmark.sty}{\usepackage{bookmark}}{\usepackage{hyperref}}
\IfFileExists{xurl.sty}{\usepackage{xurl}}{} % add URL line breaks if available
\urlstyle{same} % disable monospaced font for URLs
\hypersetup{
  pdftitle={Scoping Review Protocol: Statistical Models for Longitudinal Data in Health and Biomedical Research: Current State, Challenges, and Opportunities},
  pdfauthor={Ariel I. Mundo Ortiz},
  colorlinks=true,
  linkcolor={blue},
  filecolor={Maroon},
  citecolor={Blue},
  urlcolor={Blue},
  pdfcreator={LaTeX via pandoc}}

\title{Scoping Review Protocol: Statistical Models for Longitudinal Data
in Health and Biomedical Research: Current State, Challenges, and
Opportunities}
\author{Ariel I. Mundo Ortiz}
\date{2022-09-15}

\begin{document}
\maketitle
\ifdefined\Shaded\renewenvironment{Shaded}{\begin{tcolorbox}[sharp corners, interior hidden, frame hidden, borderline west={3pt}{0pt}{shadecolor}, boxrule=0pt, enhanced, breakable]}{\end{tcolorbox}}\fi

\renewcommand*\contentsname{Table of contents}
{
\hypersetup{linkcolor=}
\setcounter{tocdepth}{3}
\tableofcontents
}
\hypertarget{notes}{%
\section{Notes}\label{notes}}

\textbf{As of Sept 7, 2022 this document follows the structure
recommended by PRISMA-P}
https://prisma-statement.org/documents/PRISMA-P-checklist.pdf

\textbf{Scoping review is exploratory, can be a little broad but is best
to start with one to make sure that the method works, and that its not
too biased because of dispairing standards within subfields. Oncology,
neurodevelopment, mental health: psichology, psychyatry}

\begin{itemize}
\tightlist
\item
  Oncology can be the sandbox.
\end{itemize}

\hypertarget{registration}{%
\section{Registration}\label{registration}}

This section will be populated with the registration number and registry
name once the protocol is submitted for peer review.

\hypertarget{author-contributions}{%
\section{Author Contributions}\label{author-contributions}}

\begin{itemize}
\tightlist
\item
  AM: Writing, query design, data extraction and analysis . . .
\end{itemize}

Other authors to add later

\hypertarget{amendements}{%
\section{Amendements}\label{amendements}}

Protocol amendments resulting from peer review will be indicated in this
section indicating the date of each amendment.

\hypertarget{support}{%
\section{Support}\label{support}}

This section will indicate the sources of financial or other support for
the review

\hypertarget{sources}{%
\subsection{Sources}\label{sources}}

\hypertarget{introduction}{%
\section{Introduction}\label{introduction}}

\hypertarget{rationale}{%
\subsection{Rationale}\label{rationale}}

Longitudinal studies are frequently used in the health sciences
(biomedical research, epidemiology, public health, among others) to
examine the temporal effect of a treatment or intervention \textbf{add
about those studies where there is not an intervention, but follow
up/evolution}\textsuperscript{1,2}. However, the statistical analysis of
longitudinal data requires to take into consideration factors such as
data missingness, correlation, and non-linear
trends\textsuperscript{3,4}, which represent an ``analytic cost''
associated with the complexity of longitudinal data\textsuperscript{2}.

One of the problems derived from the ``analytic cost'' of longitudinal
data pertains the misspecification of the statistical models used to
analyze such data (i.e., the use of models that are not coherent with
the data), a problem that has been shown to occur frequently in the
health sciences\textsuperscript{5}. This problem with model
misspecification can be linked to a historical preference by researchers
to use the repeated measures analysis of variance (rm-ANOVA) as the
default method to analyze longitudinal data, despite the fact that the
multiple assumptions required by this model are frequently not satisfied
by the data collected in longitudinal studies\textsuperscript{4}.

On the other hand, multiple modern statistical models were developed
during the last 30 years to address the limitations of rm-ANOVA. Linear
mixed models, generalized additive mixed models, and generalized
estimating equations are among these modern statistical models developed
for longitudinal data\textsuperscript{6--10}. However, the use of such
modern statistical methods has been the exception rather than the norm
in the health sciences\textsuperscript{11}, even on this day and age
where these modern methods have been brought to a wider audience with
the development of computational tools such as Python or R.

Unfortunately, the misuse and lack of reproducibility of statistical
analyses continue to be major problems in the health
sciences\textsuperscript{12--15}. In the case of longitudinal data,
where modern methods exist beyond rm-ANOVA that can help researchers
obtain better inference from their data, there is a need to understand
what are the trends in the adoption of these statistical methods in the
health sciences to measure the adoption of reproducibility practices by
the field at large, while also identifying the reasons that may cause
researchers use avoid the use of modern statistical methods for
longitudinal data.

\hypertarget{objectives}{%
\section{Objectives}\label{objectives}}

This study aims to:

\begin{itemize}
\item
  Identify the different statistical models for longitudinal data that
  are used in the health sciences in order to measure the current extent
  in the adoption of modern statistical methods by the field (Aim 1a)
\item
  Summarize the computational tools used by researchers in the health
  sciences to statistically analyze longitudinal data to understand the
  current status of the field with regards to reproducibility. (Aim 1b)
\item
  List statistical methods for longitudinal data developed within the
  last decade in order to showcase newer methods that may be applicable
  for longitudinal data in a biomedical/health context. (Aim 2)
  \textbf{maybe a different database different from Web of Science?
  Database for Stats or Math?}
\end{itemize}

\hypertarget{review-question}{%
\section{Review Question}\label{review-question}}

\begin{itemize}
\item
  What are the statistical methods used in biomedical/health sciences
  research?
\item
  Has the use of modern statistical methods increased in the field
  during the last 20 years?
\item
  What computational tools are most commonly used by researchers to
  analyze longitudinal data, and how in turn this affects
  reproducibility?
\item
  What are most recent statistical methods developed for longitudinal
  data, and how can they be applied in the health sciences?
\end{itemize}

\hypertarget{methods}{%
\section{Methods}\label{methods}}

\hypertarget{types-of-studies}{%
\subsection{Types of Studies}\label{types-of-studies}}

For all the study aims, studies included in the analysis correspond to
peer-reviewed publications in English.

\hypertarget{eligibility-criteria}{%
\subsection{Eligibility Criteria}\label{eligibility-criteria}}

\hypertarget{for-the-application-of-modern-statistical-models-on-longitudinal-biomedicalhealth-data-aims-1a-and-1b}{%
\subsubsection{For the Application of Modern Statistical Models on
Longitudinal Biomedical/Health Data (Aims 1a and
1b)}\label{for-the-application-of-modern-statistical-models-on-longitudinal-biomedicalhealth-data-aims-1a-and-1b}}

\hypertarget{inclusion-criteria}{%
\paragraph{Inclusion Criteria}\label{inclusion-criteria}}

Articles that are written in English, belong to the biomedical/health
sciences fields, describe the collection and analysis of continous or
discrete longitudinal data, indicate the statistical model used to
analyze the data, and report the results of their statistical analyses.

\hypertarget{exclusion-criteria}{%
\paragraph{Exclusion Criteria}\label{exclusion-criteria}}

Cross-sectional studies, tutorials that present the application of
existing statistical methods to biomedical/health data, reviews,
meta-analyses, or systematic reviews on existing statistical methods for
longitudinal data, studies that use only descriptive statistics to
summarize/analyze the data, studies that collect and analyze categorical
data. \textbf{You don't want to exclude things right away, much rather
get them and then decide.}

\hypertarget{for-methods-on-longitudinal-data-aim-2}{%
\subsubsection{For Methods on Longitudinal Data (Aim
2)}\label{for-methods-on-longitudinal-data-aim-2}}

\hypertarget{inclusion-criteria-1}{%
\paragraph{Inclusion Criteria}\label{inclusion-criteria-1}}

\begin{itemize}
\tightlist
\item
  Articles that:
\end{itemize}

Are written in English, present new methodologies or significant
improvements to existing methods for longitudinal data.

\hypertarget{exclusion-criteria-1}{%
\paragraph{Exclusion Criteria}\label{exclusion-criteria-1}}

Systematic reviews, meta-analyses, or reviews of statistical methods for
longitudinal data, tutorials that present the application of existing
statistical methods to biomedical/health longitudinal data.

\hypertarget{information-sources}{%
\subsection{Information Sources}\label{information-sources}}

Studies will be retrieved from PubMed and Web of Science.

\hypertarget{search-strategy}{%
\subsection{Search Strategy}\label{search-strategy}}

PubMed and Web of Science databases will be used. Below the full search
strategy for PubMed is presented for all the aims of the scoping review.

\hypertarget{for-the-application-of-modern-models-on-longitudinal-biomedicalhealth-data}{%
\subsubsection{For the Application of Modern Models on Longitudinal
Biomedical/Health
Data}\label{for-the-application-of-modern-models-on-longitudinal-biomedicalhealth-data}}

\hypertarget{pubmed}{%
\paragraph{PubMed}\label{pubmed}}

\hypertarget{query}{%
\subparagraph{Query:}\label{query}}

(biomedical OR health) AND ((repeated measures) OR (longitudinal study)
OR (longitudinal data)) AND ((statistical analyses) OR (statistical
analysis)) NOT (Review{[}Publication Type{]} OR Meta
analy*{[}Publication Type{]}) NOT ( ``Statistics as
Topic/methods''{[}Majr{]} OR ``Statistics as Topic/statistics and
numerical data''{[}Majr{]} OR ``Models, Statistical''{[}Mesh{]} OR
``Research Design''{[}Mesh{]})

Hits: 10,972

\hypertarget{for-methods-on-longitudinal-data}{%
\subsubsection{For Methods on Longitudinal
Data}\label{for-methods-on-longitudinal-data}}

\hypertarget{pubmed-1}{%
\paragraph{PubMed}\label{pubmed-1}}

\hypertarget{query-1}{%
\subparagraph{Query 1:}\label{query-1}}

(``Models, Statistical'' {[}Mesh{]} OR
``Biostatistics/methods''{[}Mesh{]}) AND (``Longitudinal
Studies''{[}Mesh{]}) NOT (Review{[}Publication Type{]} OR Meta
Analys*{[}Publication Type{]} OR ``editorial''{[}Publication Type{]})
NOT (``survival''{[}Title/abstract{]}) NOT
(``tutorial''{[}title/abstract{]} OR
``orientation''{[}title/abstract{]}) NOT (Humans{[}Mesh{]} OR Adolescent
{[}Mesh{]} OR Animals{[}Mesh{]})

Hits: 142

\hypertarget{data-collection-and-analysis}{%
\subsection{Data Collection and
Analysis}\label{data-collection-and-analysis}}

\hypertarget{selection-process-and-data-management}{%
\subsubsection{Selection Process and Data
Management}\label{selection-process-and-data-management}}

Two reviewers will independently analyze the database search results and
pre-screen articles based on title and abstract content following the
aforementioned inclusion/exclusion criteria. Manuscripts from the
database(s) search will be stored in the Covidence platform, where
duplicated entries will be removed. For articles where pre-screening
inclusion (or exclusion) is unclear based on title and abstract
analysis, full-text review will be used to make a decision following
review by a third independent reviewer. Manuscripts included after title
and abstract pre-screening will be further screening by two reviewers
that will independently examine the full text of each article.

\hypertarget{data-collection-process}{%
\subsubsection{Data Collection Process}\label{data-collection-process}}

Pilot forms (electronic spreadsheets) will be tested using a
representative sample of the studies to be reviewed (\textasciitilde100
studies). Information in the forms will be independently included by
each reviewer. The forms will be updated (if needed), after the pilot
test by consensus between the reviewers.

Information obtained from each study (statistical method used, software,
etc.) will be tabulated independently by the reviewers in an electronic
spreadsheet.

\hypertarget{data-items}{%
\subsection{Data Items}\label{data-items}}

Aims 1a and 1b:

Statistical method used, sub-area of application (oncology, psychology,
public health, etc), computational tool used, congruence between
statistical method used and the data, year of publication

Aim 2:

Statistical method reported, assumptions of the model, computational
tools available for its implementation, year of publication

\hypertarget{risk-of-bias-in-individual-studies}{%
\subsection{Risk of Bias in Individual
Studies}\label{risk-of-bias-in-individual-studies}}

N/A

\hypertarget{data-synthesis}{%
\subsection{Data Synthesis}\label{data-synthesis}}

The data from the results of each included study will be extracted into
electronic spreadsheets. Summary measures for Aims 1a and 1b include
plots (pie, bar, etc.) to show the relative use of each statistical
method reported, computational tool, and congruence between statistical
method and the data. Each plot will be segmented by year to show trends
over time. Table~\ref{tbl-spreadsheet} presents the headers of the pilot
electronic spreadsheet.

The pilot electronic spreadsheet can be found in the following link:
\href{https://udemontreal-my.sharepoint.com/:x:/g/personal/ariel_mundo_ortiz_umontreal_ca/EUr53h3HnMBLhRohYMqYmp0Bzf1UY7ncKTvm2RkPf9Wt3w?e=cMdhdw}{Pilot Spreadsheet}

For Aim 2, a table will be created where statistical method, year of
publication, assumptions of the model, and applicability to health data
is reported.

\hypertarget{meta-biases}{%
\subsection{Meta-Biases}\label{meta-biases}}

N/A

\hypertarget{references}{%
\section{References}\label{references}}

\hypertarget{refs}{}
\begin{CSLReferences}{0}{0}
\leavevmode\vadjust pre{\hypertarget{ref-edwards2000}{}}%
\CSLLeftMargin{1. }%
\CSLRightInline{Edwards LJ. Modern statistical techniques for the
analysis of longitudinal data in biomedical research. \emph{Pediatric
Pulmonology}. 2000;30(4):330-344.
doi:\url{https://doi.org/10.1002/1099-0496(200010)30:4\%3C330::AID-PPUL10\%3E3.0.CO;2-D}}

\leavevmode\vadjust pre{\hypertarget{ref-zeger1992}{}}%
\CSLLeftMargin{2. }%
\CSLRightInline{Zeger SL, Liang K-Y. An overview of methods for the
analysis of longitudinal data. \emph{Statistics in Medicine}.
1992;11(14-15):1825-1839.
doi:\url{https://doi.org/10.1002/sim.4780111406}}

\leavevmode\vadjust pre{\hypertarget{ref-caruana2015}{}}%
\CSLLeftMargin{3. }%
\CSLRightInline{Caruana EJ, Roman M, Hernández-Sánchez J, Solli P.
Longitudinal studies. \emph{Journal of Thoracic Disease}.
2015;7(11):E537-40.}

\leavevmode\vadjust pre{\hypertarget{ref-mundo2022a}{}}%
\CSLLeftMargin{4. }%
\CSLRightInline{Mundo AI, Tipton JR, Muldoon TJ. Generalized additive
models to analyze nonlinear trends in biomedical longitudinal data using
r: Beyond repeated measures {ANOVA} and linear mixed models.
\emph{Statistics in Medicine}. Published online July 2022.}

\leavevmode\vadjust pre{\hypertarget{ref-thiese2015}{}}%
\CSLLeftMargin{5. }%
\CSLRightInline{Thiese MS, Arnold ZC, Walker SD. The misuse and abuse of
statistics in biomedical research. \emph{Biochem Med (Zagreb)}.
2015;25(1):5-11.}

\leavevmode\vadjust pre{\hypertarget{ref-pinheiro2000}{}}%
\CSLLeftMargin{6. }%
\CSLRightInline{Linear mixed-effects models: Basic concepts and
examples. In: \emph{Mixed-Effects Models in s and s-PLUS}. Springer New
York; 2000:3-56.
doi:\href{https://doi.org/10.1007/0-387-22747-4_1}{10.1007/0-387-22747-4\_1}}

\leavevmode\vadjust pre{\hypertarget{ref-jiang2021}{}}%
\CSLLeftMargin{7. }%
\CSLRightInline{Jiang J, Nguyen T. \emph{Linear and Generalized Linear
Mixed Models and Their Applications}. 2nd ed. Springer; 2021.}

\leavevmode\vadjust pre{\hypertarget{ref-hastie2017}{}}%
\CSLLeftMargin{8. }%
\CSLRightInline{Hastie TJ. \emph{Statistical Models in {S}}. (Chambers
JM, Hastie TJ, eds.). Routledge; 2017.}

\leavevmode\vadjust pre{\hypertarget{ref-rosa2004}{}}%
\CSLLeftMargin{9. }%
\CSLRightInline{Rosa GJM, Gianola D, Padovani CR. Bayesian longitudinal
data analysis with mixed models and thick-tailed distributions using
{MCMC}. \emph{Journal of Applied Statistics}. 2004;31(7):855-873.}

\leavevmode\vadjust pre{\hypertarget{ref-ballinger2004}{}}%
\CSLLeftMargin{10. }%
\CSLRightInline{Ballinger GA. Using generalized estimating equations for
longitudinal data analysis. \emph{Organizational Research Methods}.
2004;7(2):127-150.}

\leavevmode\vadjust pre{\hypertarget{ref-gueorguieva2004}{}}%
\CSLLeftMargin{11. }%
\CSLRightInline{Gueorguieva R, Krystal JH. {Move Over ANOVA: Progress in
Analyzing Repeated-Measures Data andIts Reflection in Papers Published
in the Archives of General Psychiatry}. \emph{Archives of General
Psychiatry}. 2004;61(3):310-317.
doi:\href{https://doi.org/10.1001/archpsyc.61.3.310}{10.1001/archpsyc.61.3.310}}

\leavevmode\vadjust pre{\hypertarget{ref-jarvis2016}{}}%
\CSLLeftMargin{12. }%
\CSLRightInline{Jarvis MF, Williams M. Irreproducibility in preclinical
biomedical research: Perceptions, uncertainties, and knowledge gaps.
\emph{Trends in Pharmacological Sciences}. 2016;37(4):290-302.
doi:\url{https://doi.org/10.1016/j.tips.2015.12.001}}

\leavevmode\vadjust pre{\hypertarget{ref-turkiewicz2018}{}}%
\CSLLeftMargin{13. }%
\CSLRightInline{Turkiewicz A, Luta G, Hughes HV, Ranstam J. Statistical
mistakes and how to avoid them {\textendash} lessons learned from the
reproducibility crisis. \emph{Osteoarthritis and Cartilage}.
2018;26(11):1409-1411.
doi:\href{https://doi.org/10.1016/j.joca.2018.07.017}{10.1016/j.joca.2018.07.017}}

\leavevmode\vadjust pre{\hypertarget{ref-gosselin2020}{}}%
\CSLLeftMargin{14. }%
\CSLRightInline{Gosselin R-D. Statistical analysis must improve to
address the reproducibility crisis: The {ACcess} to transparent
statistics ({ACTS}) call to action. \emph{Bioessays}.
2020;42(1):e1900189.}

\leavevmode\vadjust pre{\hypertarget{ref-lang2015}{}}%
\CSLLeftMargin{15. }%
\CSLRightInline{Lang TA, Altman DG. Basic statistical reporting for
articles published in biomedical journals: The {``statistical analyses
and methods in the published literature''} or the {SAMPL} guidelines.
\emph{Int J Nurs Stud}. 2015;52(1):5-9.}

\end{CSLReferences}

\hypertarget{tbl-spreadsheet}{}
\begin{landscape}\begin{table}
\caption{\label{tbl-spreadsheet}Pilot spreadsheet for data extraction }\tabularnewline

\centering
\resizebox{\linewidth}{!}{
\begin{tabular}{cccccccccccccc}
\toprule
\textcolor{black}{\textbf{DOI}} & \textcolor{black}{\textbf{Title}} & \textcolor{black}{\textbf{Subfield}} & \textcolor{black}{\textbf{Journal}} & \textcolor{black}{\textbf{Question}} & \textcolor{black}{\textbf{Country}} & \textcolor{black}{\textbf{\makecell[l]{Source of Result \\(Data)}}} & \textcolor{black}{\textbf{Year}} & \textcolor{black}{\textbf{\makecell[r]{Statistical\\Method}}} & \textcolor{black}{\textbf{Software}} & \textcolor{black}{\textbf{\makecell[c]{Model assumptions\\checked?}}} & \textcolor{black}{\textbf{\makecell[r]{Data/Model\\Congruency?}}} & \textcolor{black}{\textbf{Code available?}} & \textcolor{black}{\textbf{Notes}}\\
\midrule
 &  &  &  &  &  &  &  &  &  &  &  &  & \\
\bottomrule
\end{tabular}}
\end{table}
\end{landscape}



\end{document}
